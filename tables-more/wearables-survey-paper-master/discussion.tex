\section{Discussion}
Here, we discuss complementary future research directions in fields of privacy, ubiquitous computing, and user studies, along with specific limitations of this survey. 

\subsection{Future Research Directions}
Wearables are still in infancy. Perceptions of situations and capabilities will change rapidly with advancements and increased exposure. However, videos and textual information are considered to be significantly sensitive by our participants, along with past participants of smartphone user perception studies. Various systems which detect and take actions for sensitive objects in photos and videos will be critical as wearables and other devices become more ubiquitous.

While privacy and security concerns were expected, consider these self-reported concerns as inspiration for future research: the high financial costs of wearables, health concerns from constant use, safety issues related to operation, any social impacts of use, and improving device aesthetics.  

\subsection{Limitations}
One of the main limitations of this work is that our participants might not have interest,or an accurate idea of wearables and their capabilities. 83\% of our participants reported that they do not own a wearable device, but at this time, about 15\% of the general population own and use wearable devices \cite{Nilsen,WearableStatNews}, which makes our study reflective of the status quo. We believed that getting a representative survey base was a useful endeavor, although we could have easily recruited only wearables owners or people specifically interested in wearables. However, that will also have its own biases and limitations, since this does not reflect the general population. We expect user perceptions to change as rapidly as wearable technologies and the rate of adoption change. 