% THIS IS SIGPROC-SP.TEX - VERSION 3.1
% WORKS WITH V3.2SP OF ACM_PROC_ARTICLE-SP.CLS
% APRIL 2009
%
% It is an example file showing how to use the 'acm_proc_article-sp.cls' V3.2SP
% LaTeX2e document class file for Conference Proceedings submissions.
% ----------------------------------------------------------------------------------------------------------------
% This .tex file (and associated .cls V3.2SP) *DOES NOT* produce:
%       1) The Permission Statement
%       2) The Conference (location) Info information
%       3) The Copyright Line with ACM data
%       4) Page numbering
% ---------------------------------------------------------------------------------------------------------------
% It is an example which *does* use the .bib file (from which the .bbl file
% is produced).
% REMEMBER HOWEVER: After having produced the .bbl file,
% and prior to final submission,
% you need to 'insert'  your .bbl file into your source .tex file so as to provide
% ONE 'self-contained' source file.
%
% Questions regarding SIGS should be sent to
% Adrienne Griscti ---> griscti@acm.org
%
% Questions/suggestions regarding the guidelines, .tex and .cls files, etc. to
% Gerald Murray ---> murray@hq.acm.org
%
% For tracking purposes - this is V3.1SP - APRIL 2009

\documentclass{acm_proc_article-sp}

\usepackage{color}

%%% still need to break urls

\usepackage[hyphens]{url}


\def\etal{{\it et al.~}}

\newenvironment{packed_enum}{
\begin{enumerate}
  \setlength{\itemsep}{1pt}
  \setlength{\parskip}{0pt}
  \setlength{\parsep}{0pt}
}{\end{enumerate}}

\newenvironment{packed_item}{
\begin{itemize}
  \setlength{\itemsep}{1pt}
  \setlength{\parskip}{0pt}
  \setlength{\parsep}{0pt}
}{\end{itemize}}


\begin{document}

\title{Risk Perceptions for Wearable Devices}
%\titlenote{(Does NOT produce the permission block, copyright information nor page numbering). For use with ACM\_PROC\_ARTICLE-SP.CLS. Supported by ACM.}}

%\subtitle{[Extended Abstract]
%\titlenote{Note
%\textit{Note \LaTeX$2_\epsilon$\ and BibTeX} at \texttt{www.website.com}}}




%\numberofauthors{1}
%\author{
% \alignauthor Linda N. Lee\textsuperscript{1}, Serge Egelman\textsuperscript{1,3}, Joong Hwa Lee\textsuperscript{2}, David Wagner\textsuperscript{1}\\
%   \vspace{0.5em}
%   \affaddr{\textsuperscript{1}University of California, Berkeley, \{lnl,egelman,daw\}@cs.berkeley.edu}, \textsuperscript{2}dlwndghk94@berkeley.edu\\
%   \affaddr{\textsuperscript{3}International Computer Science Institute, egelman@icsi.berkeley.edu}\\
%}

\numberofauthors{1}
\author{
 \alignauthor Anonymous\\
   \vspace{0.5em}
   \affaddr{Some Place}
}

\maketitle

\begin{abstract}
We performed an online survey to examine risk perceptions surrounding wearable computing devices. We examine different data types that might be captured,  and the effect of who the data is shared with. We surveyed 1,784 participants about 72 data types and 4 data recipients to quantify risk perceptions across a wide range of scenarios, and evaluate of which factors contribute to the severity of these risks. Following previous work, we also asked participants to perform a risk/benefit analysis comparing 20 new technologies with other well-established technologies. The results of this study can be used to guide future research in wearable device security, especially research that helps to protect sensitive data and guides the design of effective privacy notifications and indicators.
\end{abstract}

% A category with the (minimum) three required fields
\category{K.6.5.}{Management of Computing and Information Systems}{Security and protection}[Unauthorized access]

%\terms{Human Factors}{Measurement}{Security}

\keywords{Privacy, Security, User Studies, Risk Perception, Ubiquitous Computing, Wearable Devices} % NOT required for Proceedings

%%%%%%%%%%%%%%%%%%%
%%%         PAPER BODY         %%%
%%%%%%%%%%%%%%%%%%%

\section{Introduction}
With their ability to constantly capture data and help users, wearable devices, or ``wearables,'' are the new frontier of ubiquitous computing. Wearable technology has many potential benefits, ranging from a more natural, human-centered interface for computing, to healthier living through fitness tracking. Forbes has named 2014 the ``Year of Wearable Technology''~\cite{Forbes}, and one of the top 25 market research companies estimates that 52\% of technology consumers are aware of wearables and among those, 33\% said they were likely to buy one~\cite{NPD}. 

The market for wearable devices is currently in its infancy, so it is hard to predict what device and applications will be most popular, but we use the current interest as a baseline. A survey of 3,956 respondents with high interest in wearables found that, currently, the most popular devices are fitness bands (61\%), followed by smart watches (45\%) and mobile health devices (17\%)~\cite{Nilsen}. It is estimated that 20\% of the general population own at least one wearable and 10\% use at least one wearable in their daily lives~\cite{WearableStatNews}. The demographics of wearables consumers are young: 48\% of owners are between 18 and 34 years old. However, it is expected that this \$700 million industry will reach other demographics~\cite{cmo}. 

Wearable devices bring with them some new potential security and privacy concerns. Many of these concerns relate to users' activities being exposed to the public without their awareness or consent. For instance, Fitbit's fitness profiles were public by default and also allowed users to track sex as an exercise~\cite{Fitbit}, resulting in the inadvertent disclosure of sensitive user information. In other instances, public discomfort prevented companies from enabling certain capabilities; Google Glass apps are prohibited from using facial recognition to mitigate potential privacy concerns~\cite{GlassDetection}.

To avoid scandalous breaches of privacy and public opposition to new capabilities, it is critical that we understand user concerns surrounding wearable devices before wearables become increasingly ubiquitous and powerful~\cite{Implants}. A better understanding of users' risk perceptions will enable researchers and companies to focus on users' greatest concerns. The goal of this paper is to gain a better sense of what those concerns might be. 

To this end, we surveyed 1,784 Internet users to determine their risk perceptions regarding wearable devices. In this work, we contribute the following: \\[-0.8cm]

\begin{itemize} \itemsep1pt \parskip0pt \parsep0pt
\item We compare users' perceptions of a wide range of privacy and security risks with wearable devices. Users care much more about the recipient of the data, than the type of data.
\item We observed that users make little distinction between sharing data with friends, co-workers, or the general public, but are relatively comfortable with an application's servers receiving their data.
\item Our participants viewed the data-collection capabilities of wearable devices as benign compared to more accepted technologies. However, we suspect that this may be due to their lack of exposure to these newer technologies.
%\item We report people's self-reported top concerns for wearable devices. Privacy, by far, is the top concern. Other notable risks include information security, long-term health risks from use, high financial cost, and change in social norms. 
\end{itemize}

%%%%%%%%%%%%%%%%%%%

\section{Methodology}

Our survey contained two main sections.
In one section, we presented participants with scenarios---something undesirable a wearable device might do---and asked them to rate their level of concern if each scenario were to happen.
This was intended to elicit their perception of the severity of the impact of the risk.
In the other section, we asked participants to compare the risks and benefits of wearable technologies with better understood technologies, following the same methodology as a seminal study in risk perception from Fischhoff \etal \cite{Fischhoff}.
Our survey design is based on two prior risk perception studies, as we describe next.

%Finally, we collected demographic information, which included a privacy concern scale, and whether participants owned any wearable devices. 

%LL: this commented out for anonymity; can comment back in later.
%%% The full survey can be found at http://www.surveygizmo.com/s3/1657924 /Wearables-Threats-User-Survey. 

%::LL: David, I spliced the related work; the text related to work directly related to our research to introduce/motivate the methodology remains here. The rest were put at the end of the paper, as you requested. 
 

\subsection{Methodology Motivation}

\subsubsection{Smartphone User Concerns}
Felt \etal previously studied the security concerns of smartphone users by conducting a large-scale online survey~\cite{Felt}. Their survey asked 3,115 smartphone users about 99 risks associated with various smartphone privileges. Participants were asked how upset they would be if a certain action had occurred without their permission. Participants rated each situation on a Likert scale ranging from ``indifferent (1)'' to ``very upset (5).''
Our methodology follows that study closely, but with a different set of scenarios that are appropriate to wearable devices.

\subsubsection{Technology Risk Perception}
Fischhoff \etal performed a seminal study of perceived risks with 30 widely used technologies~\cite{Fischhoff}. In their study, participants were asked to separately rate the risks and benefits of these technologies. They were told to think about all people affected by the technology, and to think about long-term vs. short-term risks and benefits. Then, the participants rated these technologies with respect to each other on a numerical scale, being instructed to rate the least risky or least beneficial technology a 10 and scaling the ratings linearly (e.g., a technology with risk rating 20 would be considered twice as risky compared to a technology with a risk rating of 10).
We apply their methodology to evaluate perceived risks and benefits of several technologies related to wearable computing.


%Using this methodlogy, they were able to categorize different technologies based on whether they were high-risk/high-benefit, low-risk/low-benefit, and so forth. 

\subsection{Survey Questions}
In our survey, each participant answered 27 questions, across five different sections:   \\[-.8cm]

\begin{itemize} \itemsep1pt \parskip0pt \parsep0pt
\item 2 comprehension questions
\item 6 questions about wearable computing scenarios 
\item 2 questions about smartphone scenarios 
\item 2 risk/benefit questions 
\item 15 demographic questions \\[-.8cm]
\end{itemize}

We randomized the order in which participants completed each section of the survey (with the exception of the comprehension and demographic questions, which were always first and last, respectively), as well as the order of the questions within each section.

\subsubsection{Comprehension Questions}
Because participants might be biased to specific devices or companies (e.g., visceral reactions to Google Glass based on popular media stories), we based our questions on a fictitious wearable device. Thus, the beginning of the survey introduced participants to the ``Cubetastic3000,'' which was the basis for all questions on wearable computing risks. We highlighted the capabilities of this device and described several use cases. To ensure that participants had read and understood this device's capabilities, we presented them with two multiple-choice comprehension questions.

\subsubsection{Wearables Scenarios}
We presented participants with scenarios involving data capture using the Cubetastic3000 and asked them to rate how upset they would be if a particular data type (e.g., video, audio, gestures, etc.) were shared with a particular data recipient without asking them first. All responses were collected on a 5-point Likert scale (from ``indifferent'' to ``very upset''), which was modeled after Felt et al.'s study of smartphone users' risk perceptions~\cite{Felt}. Questions were of the format, \textit{``How would you feel if an app on your Cubetastic3000 learned <data> and shared it with <recipient>, without asking you first?''}. We created an initial pool of 288 questions by combining 72 data types (<data>) with 4 data recipients (<recipient>). The 4 possible data recipients were:  \\[-.8cm]

\begin{packed_item}
\item Your work contacts
\item Your friends
\item The public
\item The app's server (but didn't share it with anyone else)
\end{packed_item}

The main purpose of these questions was to determine the extent to which the data type and data recipient played a role in upsetting participants when data is inappropriately shared. Additionally, we added 16 questions about other misbehaviors that did not follow this format, lacking either <data> or a <recipient>, but we thought were relevant nonetheless. An example of one of these questions was, ``\textit{How would you feel if an app on your Cubetastic3000 turned your device off, without asking you first?}'' Thus, there were a total of 304 questions in this set, from which we randomly presented each participant with 6 questions.

\subsubsection{Smartphone Scenarios}
\label{sec:smartphones}
We presented participants with a second set of scenarios in order to control for the type of device being used. These questions followed the format of the previous question set, but substituted ``smartphone'' for ``Cubetastic3000.'' Rather than using the previous pool of 304 <data> and <recipient> combinations, we created 5 questions based on the scenarios that Felt \etal found least and most concerning to their participants~\cite{Felt}. We randomly presented participants with 2 of the following 5 questions: \\[-.8cm]

\begin{packed_enum}
\item \textit{How would you feel if an app on your smartphone vibrated your phone without asking you first?}
\item \textit{How would you feel if an app on your smartphone connected to a Bluetooth device (like a headset) without asking you first?}
\item \textit{How would you feel if an app on your smartphone un-muted a phone call without asking you first?}
\item \textit{How would you feel if an app on your smartphone took screenshots when you were using other apps, without asking you first?}
\item \textit{How would you feel if an app on your smartphone sent premium (they cost money) calls or text messages, without asking you first?} 
\end{packed_enum}

\subsubsection{Risk and Benefit Assessment}
In addition to investigating reactions to particular scenarios, we also examined broad perceptions of various new technologies, and how those perceptions compared to other well-established technologies. To this end, we modeled this section of the survey after the seminal risk perception study by Fischhoff \etal\cite{Fischhoff}, in which they asked participants to relatively rank several technologies by both their risk and benefit to society. We asked participants to perform this exercise for 4 technologies previously examined by Fischhoff {\it et al.}: handguns, motorcycles, lawnmowers, and electricity. We chose these technologies because they represented varying levels of risks and benefits.

Alongside the 4 well-established technologies, we asked participants to evaluate one of 20 possible technologies relevant to wearable computing: the Internet, email, laptops, smartphones, smart watches, fitness trackers, Google Glass, Cubetastic3000, discrete camera, discrete microphone, facial recognition, facial detection, voice recognition, voice-based emotion detection, location tracking, speech-to-text, language detection, heart rate detection, age detection, and gender detection. Note that we ask about familiar new technologies such as the internet, general classes and specific artifacts of wearable devices, and a range of new capabilities. 

%LL: we didn't pick it because they were studied by this study, so this is a bit distracting, especially since the related work is at the end instead of before this.. 
%Many of these technologies were selected from those studied by Egelman \etal\cite{Egelman2015}.

To parallel Fischhoff {\it et al.}'s risk perception study, we gave our participants a similar prompt to numerically express the perceived gross risk/gross benefit over a long period of time for all parties involved. We randomized whether they performed the ranking for risks or benefits first. The prompt is listed in Appendix \ref{sec:prompt}. The question format was as follows:

\textit{Fill in your <risk/benefit> numbers for the following:}\\[-.5cm]

\textit{Handguns}: \_\_\_\_\_\_\_ \\
\textit{Motorcycles}: \_\_\_\_\_\_\_\\
\textit{Lawnmowers}: \_\_\_\_\_\_\_\\
\textit{<Wearable Technology>}: \_\_\_\_\_\_\_\\
\textit{Electricity}: \_\_\_\_\_\_\_\\ [-.5cm]

\subsubsection{Additional Questions}
The exit portion of the survey consisted of standard demographics questions such as age, gender, and education. We also asked participants if they owned a wearable device so that we could control for prior exposure, and included an open-ended question on what would be the most likely risks associated with wearable devices. Finally, we included the 10-question Internet Users' Information Privacy Concerns (IUIPC) index~\cite{malhotra2004internet}, so that we could control for participants' general privacy attitudes.

\subsection{Focus Group}
We conducted a one-hour focus group to validate our design, gauge comprehension, and measure fatigue. The focus group began with participants taking the survey. Afterward, we asked participants to give feedback on the format and the content, noting any instructions or questions that were unclear. The focus group concluded with a discussion of possible benefits and risks of wearable devices, in order to brainstorm any additional scenarios to include. Finally, we compensated participants with \$30 in cash. We recruited all of our focus group participants from Craigslist. Of the 13 participants, 54\% were female, and ages ranged from 18 to 64 (mu = 36.1, sigma = 15.3).  Education backgrounds ranged from high school to doctorate degrees, and professions included student, artist, marketer, and court psychologist.

\subsection{Recruitment and Analysis Method}
We recruited 2,250 participants August 7th-13th 2014 via Amazon's Mechanical Turk. We restricted participants to those over 18 years old, living in the United States, and having a successful HIT completion rate of 95\% or above. Based on incorrect responses to either of the two comprehension questions, we filtered out 366 (16\% of 2,250) participants. We filtered out an additional 99 participants (4\% of 2,250) due to incomplete responses, and one participant who was under 18, leaving us with a total sample size of 1,784. Of these, 55.10\% were male, with a median age of 29 ($\sigma$ = 10.37).

In performing our analysis in the next section, we chose to focus on the very upset rate (VUR) of various scenarios (i.e., the percentage of participants who reported a `5' on the Likert scales), rather than the average of all Likert scores. We did this for the same reasons as Felt {\it et al.}: the VUR does not presume that the ratings, ranging from ``indifferent'' to ``very upset,'' are linearly spaced. Additionally, we believe that everyone would be upset, at least a little, at the given scenarios (i.e., in all scenarios, the device takes action without permission). Thus, we believe that the main distinguishing factor of a participant reacting to a given scenario is if they were maximally upset or not, rather than how upset they were. In the risk/benefit section, we followed Fischhoff {\it et al.}'s methodology and did not normalize the numerical responses. Rather, we report medians and quartiles, which are not heavily impacted by outliers. Finally, for the open-ended question at the end (i.e., additional privacy concerns), two researchers independently coded 1,784 responses and had an initial agreement rate of 89.7\%. The researchers discussed any disagreements, and resolved them so that the final codings reflect unanimous agreement.

%%%%%%%%%%%%%%%%%%%

\section{Results}
In this section, we present our survey results. We first discuss participants' responses to the various data-sharing scenarios, and how data type, device type, and recipient contributed to how threatening a situation was perceived. Next, we discuss participants' risk/benefit assessment of various new technologies relative to well-established technologies. We conclude the section with participants' self-reported concerns about the biggest risks in owning wearable devices.

\subsection{Concern Factors}
In this section, we discuss the factors impacting participants' concern levels for each scenario: the data recipient, the data type, and whether or not the scenario occurred on a wearable device or a smartphone. We analyze each factor individually, as well as present a statistical model of participants' concerns as a function of all of these factors, including demographic traits.

\subsubsection{Data Recipient}
Based on our data, we observed that the largest effect seemed to stem from who the recipient of the data was: across all scenarios, 42.3\% of participants stated that they would be ``very upset'' if their data was shared with only the app's servers, whereas the VURs for friends (69.5\%), work contacts (75.2\%), and the public (72.4\%) were much higher. A chi-square test indicated that these differences were statistically significant (Table \ref{recipient}). However, these effect sizes were relatively small: the largest effect was between work contacts and an app's server ($\phi=0.11$); while the VUR for sharing with work contacts was significantly higher than sharing with friends, the effect size was negligible ($\phi=0.004$).

We note that this chi-square test violates the assumption of independence of observations, since each participant responded to multiple scenarios. However, based on the randomization of treatments and very large sample size, we do not believe that this impacted our results. Nonetheless, we repeated the analysis using only one randomly-selected data point per participant and found that the test was robust to this violation: participants were significantly more concerned about having their data seen by actual humans ({\it vis-{\`a}-vis} app servers), though differences between specific human groups were not significant (i.e., between the public, friends, and work contacts).

%Comment on differences between public/work, and end the subsection
%SE: this needs to be rewritten below.

We compared the 10 most concerning scenarios when sharing with an app's servers versus with a human audience (friends, work contacts, and the public). We observed that there was a substantial overlap between these groups, in that 6 of the most concerning scenarios were the same:

\begin{packed_enum}
\item Bank account information
\item A video of you unclothed
\item Social security number
\item Video of you entering your PIN
\item An incriminating/embarrassing photo of you
\item A photo of you unclothed
\end{packed_enum}

What is interesting about this finding is that while the concerning data types do not appreciably change based on the data recipient---even the non-overlapping scenarios all dealt with confidential data (e.g., passcode, credit card information, etc.)---only the level of concern changed. For instance, the 10th most concerning scenario for the non-human audience had a VUR of 66.67\%, whereas the 10th most concerning scenario for a human audience has a VUR of 93.88\%. This suggests that concern for different data types does not appear to vary relative to other data types based on recipient, but instead the recipient determines the overall magnitude of the concern.

%\begin{table}[t]
%\begin{center}
%\begin{tabular}{|l|r|}
%\hline
%Recipients & VUR \\
%\hline
%App &  42.3\% \\
%Friends & 69.5\% \\
%Work & 75.2\% \\
%Public & 72.4\% \\ 
%\hline
%\end{tabular}
%\caption{generated for my presentation.}
%\label{nouse}
%\end{center}
%\end{table}


\begin{table}[t]
\begin{center}
\begin{tabular}{|l|r|r|r|r|}
\hline
Recipients	& $\chi^2$ & p-value 	& n & $\phi$ \\
\hline
Work-App	& 565.910 & <0.0001 & 5,083 & 0.111\\
Public-App	& 481.776 & <0.0001 & 5,1988& 0.093\\
Friends-App & 381.653 & <0.0001 & 5,096 & 0.075\\
Friends-Work & 20.39 & <0.0001 & 5,037 & 0.004\\
Friends-Public & 5.41 & <0.0200 & 5,142 & 0.001\\
Work-Public&  5.00 & <0.0253 & 5,129	& 0.001\\
\hline
\end{tabular}
\caption{Results of a chi-square test to examine VUR based on data recipient, across all data points.}
\label{recipient}
\end{center}
\end{table}

%\begin{table}[t]
%\begin{center}
%\begin{tabular}{|l|r|r|r|r|}
%\hline
%Recipients	& $\chi^2$ & p-value 	& n & $\phi$ \\
%\hline
%Work-App	& 42.49	& <0.0001	&	601	&	0.071\\
%Public-App &	48.52	& <0.0001	&		636	&	0.076\\
%Friends-App	& 32.07	& <0.0001	&		609	&	0.053\\
%Friends-Work & 	0.87 &	<0.3517	&	604	&	0.001\\
%Friends-Public	& 1.46 &	<0.2229	&	639	&	0.002\\
%Work-Public &	0.67 &	<0.7956	&		631	&	0.001\\
%\hline
%\end{tabular}
%\caption{Results of a chi-square test to examine VUR based on data recipient, across all data points.}
%\label{recipient}
%\end{center}
%\end{table}

%\begin{table}[t]
%\begin{center}
%\begin{tabular}{| c | c | c | c |}
%Recipients	& $\chi^2$ &	2-tail P &  Effect Size \\
%Work-App	& 564.318 & <0.0001 & 0.111\\
%Public-App	& 479.980 & <0.0001 &  0.092\\
%Friends-App & 380.000 & <0.0001 & 0.075\\
%Friends-Work & 20.365 & <0.0001 &  0.004\\
%Friends-Public & 5.349 & 0.0207 &  0.001\\
%Work-Public&  5.054 & 0.0246 &  0.001\\
%\end{tabular}
%\caption{Chi-Squared test results of the effects of various recipients contributing to VUR.}
%\label{recipient}
%\end{center}
%\end{table}				

%\begin{table}[t]
%\begin{center}
%\begin{tabular}{| l | c |}
%<data> & VUR  \\
%social security number & 98.04\% \\ CC
%a video of you unclothed & 97.44\% \\ CC
%bank account information & 97.10\% \\ CC
%recordings of your work conversations & 96.97\% \\ 
%an incriminating/embarrassing photo of you & 96.36\% \\ CC
%a photo of you unclothed & 96.30\% \\ CC
%credit card information & 95.92\% \\ 
%username and password for websites & 95.41\% \\ 
%a video of you entering in your PIN & 93.91\% \\ CC
%recordings of your phone conversations & 93.88\% \\ 
%\end{tabular}
%\caption{A table of the top 10 upsetting <data> types, with respect to <recipient> types friends, work, and public.}
%\label{sharedtop10}
%\end{center}
%\end{table}
%
%\begin{table}[t]
%\begin{center}
%\begin{tabular}{| l | c |}
%%starting from #63
%<data> &  VUR  \\
%bank account information & 90.91\% \\ CC
%a video of you unclothed & 90.62\% \\ CC
%social security number  & 88.68\% \\ CC
%video of you entering your PIN & 88.57\% \\ CC
%an incriminating/embarrassing photo of you & 78.05\% \\ CC
%a photo of you unclothed & 77.78\% \\ CC
%a video of you entering a passcode to a door & 75.00\% \\ 
%when and how much you have sex & 73.08\% \\ 
%an incriminating/embarrassing video of you & 71.88\% \\ 
%a random (inward-facing) photo of you at home & 66.67\% \\
%\end{tabular}
%\caption{A table of the top 10 upsetting <data> types, respect to <recipient> app's server only (didn't share it with anyone else).}
%\label{notsharedtop10}
%\end{center}
%\end{table}
%
%\begin{table}[t]
%\begin{center}
%\begin{tabular}{| l | c |}
%%starting from #63
%<data> &  VUR  \\
%your name & 47.25\% \\
%when and how much you exercise & 46.07\% \\
%when you were happy or having fun & 38.10\% \\
%what television shows you watch & 35.96\% \\
%when you are busy or interruptible & 34.34\% \\
%your heart rate & 32.28\% \\
%music from your device & 31.87\% \\
%your age & 29.67\% \\
%the language you speak & 20.95\% \\
%your gender & 16.81\% \\
%\end{tabular}
%\caption{A table of the bottom 10 upsetting <data> types, with respect to <recipient> types friends, work, and public.}
%\label{sharedbottom10}
%\end{center}
%\end{table}
%
%
%\begin{table}[t]
%\begin{center}
%\begin{tabular}{| l | c |}
%%starting from #63
%<data> &  VUR  \\
%when and how much you exercise & 16.67\% \\
%how much you use your phone & 15.79\% \\
%your age & 14.29\% \\
%how much you like the people you interact with & 13.79\% \\
%when, what, and how much you ate & 12.50\% \\
%which television shows you watch & 11.43\% \\
%your gender & 9.52\% \\
%your heart rate & 9.09\% \\
%eye movement patterns (for eye tracking) & 6.98\% \\
%the language you speak & 2.50\% \\
%\end{tabular}
%\caption{A table of the bottom 10 upsetting <data> types, respect to <recipient> app's server only (didn't share it with anyone else).}
%\label{notsharedbottom10}
%\end{center}
%\end{table}
%
%Commonly, bank info, SSN, PIN, embarrassing photo, naked photo, naked video were considered to be highly sensitive. When people perceived that the data would be shared with the app only, the other top five concerns included the passcode to a door, how frequently one has sex, embarrassing videos, and photos at home. When people perceived that the data would be shared with a human audience, their other top concerns were work conversations, credit card information, username and password combinations for websites, and phone conversations. When data is perceived to be shared by an app, people are more concerned with issues of being spied on or tracked, whereas when data is perceived to be shared with an individual, people are concerned more with theft or reputation. 
%
%On the other hand, exercise details, age, tv shows, gender, heart rate, and language were commonly considered to be lease concerning. When people perceived that the data would be shared with the app only, the other indifferent data included phone use, how much one liked the people around, when and what you ate, and eye patterns. When people perceived that the data would be shared with a human audience, the least concerning data included one's name, if one was having fun, music on the device, and if one was busy or interruptible. When sharing with an application, information otherwise considered personal such as phone use, opinions of people, food eaten, or eye patterns were okay to share, since these seem like useful information to improve one's device experience. However, people are not likely to share the same with people, but are more comfortable with sharing data about topics which would come up in causal conversation.

\subsubsection{Data Type}
Regardless of the data recipient or the device, participants were most concerned about photos and videos, especially if they contained embarrassing content, nudity, or financial information. Information that could be used to impersonate someone (e.g., usernames/passwords for websites) or invade privacy (photos of someone at home) were also among the most concerning data types. The 10 most and least concerning data types can be seen in Table \ref{top10}.

The least concerning data types were all related in that they all consisted of information that could be observed through observations of public behavior. For instance, demographic information (e.g., age, gender, language spoken). It is possible that people rated some of these as unconcerning because they believed that many entities already track this data (e.g., shows watched, music listened to, exercise patterns).

\begin{table}[t]
\begin{center}
\small
\begin{tabular}{| r | l | r |}
\hline
Rank & Data &  VUR  \\
\hline
1 & a video of you unclothed & 95.97\% \\
2 & bank account information & 95.91\% \\
3 & social security number & 94.84\% \\
4 & video of you entering in your PIN & 92.67\% \\
5 & a photo of you unclothed & 92.59\% \\
6 & an incriminating/embarrassing photo of you & 91.39\% \\
7 & username and password for websites & 89.55\% \\
8 & credit card information & 88.98\% \\
9 & an incriminating/embarrassing video of you & 88.41\% \\
10 & a random (inward-facing) photo you at home & 87.50\% \\
 & \vdots & \\
64 & eye movement patterns (for eye tracking) & 40.51\% \\
65 & when and how much you exercise  & 38.66\% \\
66 & when you are happy or having fun  & 34.75\% \\
67 & which television shows you watch & 30.20\% \\
68 & when you are busy or interruptible  & 29.50\% \\
69 & music from your device  & 28.06\% \\
70 & your heart rate & 27.50\% \\
71 & your age & 24.29\% \\
72 & the language you speak & 15.86\% \\
73 & your gender & 15.00\% \\ 
\hline
\end{tabular}
\caption{The 10 most and least upsetting data types, across all recipients.}
\label{top10}
\end{center}
\end{table}

%A statistical analysis regarding the significance and confidence of <data> types with respect to all 72 was not performed due to the space constraints of the paper. We do consider all <data> categories in our statistical model, which provides an analysis of what factors had contributed to the perceived severity of a particular situation. 

\subsubsection{Device Type}
Participants had unique VUR rates for situations only differing in the device. Our participants had a 58.79\% VUR when asked about wearables devices and 46.64\% VUR when asked about smartphones.The VUR rates for both devices for all 5 questions are in table ~\ref{deviceVUR}. However, the effect the device has on the VUR is not considered to be statistically significant (see Table ~\ref{betweendevice}). Additionally, there is no statistically significant difference between how people reacted in a given situation; although, participants were statistically significantly upset in Q2.  The aforementioned results are only with respect to between subjects analysis, where answers are from participants who received either only the wearables or smartphone version of the 5 questions. There were too few instances where a participant got both versions of questions (34 in total for all 5 questions), to be able to perform a within-subjects analysis. 

\begin{table}[t]
\begin{center}
\begin{tabular}{| c | r | r |}
\hline
 Question &  Wearable VUR & Smartphone VUR \\
 \hline
 All & 58.79\% & 46.64\%\\
Q1 & 14.81\%  &  6.13\%\\
Q2 & 44.11\%  &  19.85\%\\
Q3 & 87.09\%  &  58.44\%\\
Q4 & 52.77\%  & 55.74\%\\
Q5 & 86.49\%  &  91.82\%\\ 
\hline
\end{tabular}
\caption{VURs for the questions described in Section \ref{sec:smartphones}, contrasting smartphones with the Cubetastic3000.}
\label{deviceVUR}
\end{center}
\end{table}

\begin{table}%[h]
\begin{center}
\begin{tabular}{|l|r|r|r|r|}
%Question & Chi^2 &	2-tail P &	Sig?	 & n	& effect size\\
%All	2.814	0.1395	No		3589		0.001\\
%Q1	2.500	0.1139	No		714		0.004\\
%Q2	17.333	<0.0001	Yes		708		0.024\\
%Q3	0.020	0.8886	No		699		0.000\\
%Q4	1.426	0.2324	No		731		0.002\\
%Q5	1.611		0.2043	No		710		0.002\\
\hline
Question & $\chi^2$ & p-value & n & $\phi$ \\
\hline
All & 2.202 & <0.1378 & 3,588 & 0.001\\
Q1 & 2.500 & <0.1139 & 714 & 0.004\\
Q2 & 17.333 & <0.0001 & 708 &  0.024\\
Q3 & 0.020 & <0.8886 & 699 &  0.000\\
Q4 & 1.413 & <0.2345 & 730& 0.002\\
Q5 & 1.604 & <0.2054 & 709 & 0.002\\
\hline
\end{tabular}
\caption{Chi-square test results comparing participants' VURs between the smartphone and Cubetastic3000 questions.}
\label{betweendevice}
\end{center}
\end{table}
						
%\begin{table}%[h]
%\begin{center}
%\begin{tabular}{| c | c | c | c |}
%Question & $\chi^2$ &	2-tail P & Effect Size \\
%All & 0.444  & 0.505 & 0.006\\
%Q1 & 0 & 1 & 0 \\
%Q2 & 0 & 1 & 0\\
%Q3 & 0 & 1 & 0\\
%Q4 & 0 & 1 & 0\\
%Q5 & 0 & 1 & 0\\
%\end{tabular}
%\caption{Results of the effects of <device> type contributing to VUR. Values are McNeamar's test results for within-subjects comparisons for participants who received both questions.}
%\label{withindevice}
%\end{center}
%\end{table}

%For the between-subjects comparison (i.e., participants who received one question, but not the other), you can do either a chi-square test or Fisher's exact test. Both use a 2 x 2 contingency matrix (i.e., rows are outcomes---upset or not---and columns are conditions---wearable or smartphone). The way to choose between the two tests is based on sample size, generally you use chi-square when there are more than 10-20 samples per cell in the table, but using either is just as valid. For the within-subjects comparisons (i.e., participants received both questions), you would do McNemar's test; which is the within-subjects version of the chi-square test. 

\subsubsection{Demographic Factors}

We observed that participants' responses to our scenarios were correlated with several demographic factors. First, we observed that by far the biggest factor in participants' decisions to rate a scenario as very upsetting was their self-reported general privacy concern level, as determined by the IUIPC scale~\cite{malhotra2004internet}: a Spearman correlation yielded a large statistically significant effect when comparing average IUIPC scores with the fraction of scenarios that each participant reported as being very upsetting ($\rho=0.446$, $p<0.0005$). Similarly, we observed that age was also a significant predictor of VUR ($\rho=0.121$, $p<0.0005$). We suspect that the age effect is due to the significant correlation between age and IUIPC scores ($\rho=0.188$, $p<0.0005$), as others have also observed that older individuals tend to be more protective of their privacy~\cite{varian2005demographics}.

While we initially observed an effect on VURs based on whether or not participants claimed to already own wearable devices (57.0\% vs. 60.8\%, respectively; Mann-Whitney $U=202,896$, $p<0.032$), this difference did not remain significant upon correcting for multiple testing (Bonferroni corrected $\alpha=0.01$), nor did the effect of gender. Finally, we observed no correlation between education level and VUR.

\subsubsection{Regression Model} 
In order to examine the relative effect of each factor on participants' VURs, we constructed several statistical models to predict whether a participant would be ``very upset'' with a given scenario based on the data type, device, data recipient, and their demographic factors (i.e., age, education, gender, and privacy attitudes). We performed binary logistic regressions using generalized estimating equations, which account for our repeated measures experimental design (i.e., each participant contributed multiple data points).

\begin{table}
\centering
\begin{tabular}{|l| r| r| r|}
\hline
Parameters & $\chi^2$ & $df$ & QIC\\
\hline
\hline
(Intercept) & 255.0 & 1 & 18,477.5\\
\hline
(Intercept) & 78.4 & 1 & 18,122.9\\
Device Type & 400.3 & 1 & \\
\hline
(Intercept) & 289.1 & 1 & 17,667.5\\
IUIPC (covariate) & 368.5 & 1 & \\
\hline
(Intercept) & 297.8 & 1 &17,383.6\\
Data Recipient & 913.4 & 4 & \\
\hline
(Intercept) & 374.6 & 1 & 14,794.5\\
Data Type & 1,866.5 & 77 & \\
\hline
(Intercept) & 303. & 1 & 13,942.9\\
Device Type & 11.1 & 1 &  \\
Data Recipient & 624.6 & 3 & \\
Data Type & 1,961.2 & 76 &  \\
\hline
(Intercept) & 28.4 & 1 & 12,752.8 \\
Device Type & 8.8 & 1 &  \\
Data Recipient & 577.8 & 3 & \\
Data Type & 1,752.1 & 76 & \\
IUIPC (covariate) & 378.7 & 1 & \\
\hline
\end{tabular}
\caption{Goodness-of-fit metrics for various binary logistic models of our data using general estimating equations to account for repeated measures. The columns represent the Wald test statistic for each parameter and the overall Quasi-Akaike Information Criterion (QIC) for each model.}
\label{regression}
\end{table}

We created several models using our three dependent variables as factors: device type (smartphone vs. wearable), data recipient, and data type. We also used our collected demographic factors as covariates: age, gender, education, wearable device ownership (yes/no), and mean IUIPC score. For each model, we performed Wald's test to examine the model effects attributable to each of these eight parameters and observed that the only covariate that had an observable effect on our models was participants' IUIPC scores. Thus, we opted to remove the other covariates from our analysis.

Table \ref{regression} shows the various models that we examined and the Quasi-Akaike Information Criterion (QIC), which is a goodness-of-fit metric for model selection (lower relative values indicate better fit). As can be seen, while the remaining four predictors all contributed to the predictive power of our model, the data type was the strongest predictor. Conversely, despite being significant, the device type was the weakest predictor (i.e., whether participants were answering questions about a smartphone or a wearable device).


\subsection{Risk and Benefit Rankings} 

When assessing new technologies, participants generally rated all new technologies and new wearables similarly, considering them low risk and low benefit (see figure \ref{fig:techplot}). We suspect that these similar assessments are because most are not consciously aware of the possibilities of these technologies or how they are being used today.


\begin{figure}
	\centering
	\includegraphics[width=0.5\textwidth]{techplot.png}
	\caption{(This is a placeholder! TODO: generate a better plot; take out the specific wearables too.)}
	\label{fig:techplot}
\end{figure}


Participants rated the the most familiar technologies as the most beneficial; 5 most beneficial technologies were internet, laptops, email, smartphones, and location tracking. Again, we believe this is the result of exposure people have to these technologies--most people are exposed to these devices and use location-based applications. The most risky technologies were ones perceived to be privacy-invasive; 5 most risky technologies were facial recognition, internet, discreet video camera, location tracking, and facial detection. People are becoming increasingly aware of such privacy risks and are comparing these privacy invasion to real physical risks--for instance, the capacity for facial detection on a wearable device is perceived to be almost as risky as interacting with a physical lawnmower. It is fair to say that, these assessments of the most risky or beneficial technologies may not be the most accurate, but do reflect the exposure of various technologies to the general public and also prove that people do perceive risks in a real way related to these technologies. 

\subsection{Perceived Concerns for Wearable Devices}
So far, participants had responded to questions regarding specific scenarios or assessed technologies presented to them. For completeness of the survey, we also wanted to capture the participants' general reactions to wearable devices as a whole. To do this, we asked the participants an open-ended question regarding the most likely risks associated with owning and interacting with wearable devices.

\textit{What do you think are the most likely risks associated with wearable devices?}\\[-.5cm]

This question was asked along with the demographics questions, near the end of the survey, but before any IUIPC questions regarding privacy, to avoid biasing. The participants were presented with a blank box to write in, with no character limit to their open-ended responses. 

Without any doubt, the open-ended responses state that the most common concern for owning and interacting with wearable devices for the average user is the \textit{possible loss of privacy}. This top concern surpassed all other concerns by about an order of magnitude or more:

\begin{table}[h]
\begin{center}
\begin{tabular}{|l|r|r|}
\hline
Concern &  Responses &  Frequency   \\
\hline
Privacy & 452 & 25.32\% \\
Being Unaware & 275 & 15.40\% \\
%Unaware Use & 167 & (9.36\%)\\
%Unaware Collection & 64 & (3.59\%)\\
%Unaware Access & 44 & (2.46\%)\\
Health Risk & 191 & 10.70\%\\
Safety & 185 & 10.42\%\\
Social Impact &	157 & 8.80\%\\
Financial Cost & 151 & 8.46\%\\
Security &	144 & 8.07\%\\
Accidental Sharing &	69 & 3.87\%\\
Miscellaneous &	57 & 3.19\%\\
None	& 51 & 2.86\%\\
Social Stigma &	39 & 2.18\%\\
False Information & 33 & 1.85\%\\
Don't know & 31 & 1.74\%\\
Aesthetics 	& 19 & 1.06\%\\
Don't care 	& 11 & 0.62\%\\
\hline
\end{tabular}
\caption{A table listing the self-reported most common risks associated with owning a wearable device.}
\label{open-responses}
\end{center}
\end{table}

Other significant secondary concerns included being unaware of what the device is collecting, doing, or which information it is using (Being Unaware), long-term health effects caused from wearing the device such as cancer from emf waves (Health), safety hazards from wearing the device such as battery burns or distractions causing car accidents (Safety), resulting changes in social behaviors, such as dependencies on devices or spending less time with loved ones (Social Impact), the high financial cost of buying, replacing, or caring for the device (Financial), and information compromise (Security). Refer people to the appendix \ref{sec:coding} for detailed information on coding labels. 

%%%%%%%%%%%%%%%%%%%

\section{Discussion}
We take this section to discuss complementary future research directions in fields of privacy, ubiquitous computing, and user studies, along with specific limitations of this survey. 

\subsection{Future Research Directions}
Wearables are still in their infancy. Perceptions of situations and new capabilities will change rapidly with new capabilities and increased exposure, respectively. However, videos, and textual information is considered to be significantly sensitive by our participants, along with past participants of smartphone user perception studies. Various systems which can detect and take actions for sensitive objects in photos and videos will be critical as wearables and other devices become more ubiquitous.

While privacy and security concerns were expected, consider these unexpected user concerns as starting points for future research: the high financial costs of wearables, health concerns of constant device use, safety issues related to device operation, the social impact of wearable devices, and improving aesthetics of wearables.  

\subsection{Limitations}
One of the main limitations of this work is that our participants might not have interest, or an accurate idea, of wearables and their capabilities. 83\% of our participants reported that they do not own a wearable device, but at this time, about ~15\% of the general population own and use wearable devices \cite{Nilsen,WearableStatNews}, which makes our study reflective of the status quo. We believed that getting a representative survey base was a useful endeavor, although we could have easily recruited only wearable device owners or people specifically interested in wearables. However, that will also have its own bias and limitations as well, since they would not reflect the general population. We expect user perceptions to change as rapidly as wearable technologies and the rate of adoption change. 

% LL: David, thanks for polishing up the related work!
%LL: Related work put at the end upon David's Suggestion. Paper flow is much better. 
\section{Related Work}
{\color{red} this section is still rough! David will redo.} In this section, we discuss related work on emerging challenges related to ubiquitous computing, threats to smartphone users, and risk perceptions pertaining to technologies.

\subsection{Ubiquitous Sensing}
Due to new technological advances, we are rapidly moving towards ubiquitous sensing and data capture~\cite{abowd2000charting}, whether it be through smartphones, smarthomes, or miscellaneous networked devices. Many have worked to discover how privacy will be preserved in world of nearing omni-present sensors. Examples of such efforts include frameworks to design for privacy environments~\cite{bellotti1993design} and evaluating ubiquitous computing applications~\cite{scholtz2004toward}. There are also various models for understanding privacy with respect to ubiquitous computing, including a risk-based model~\cite{hong2004privacy} and a social-based model~\cite{jiang2002approximate}. 

Wearables are the new frontier of ubiquitous computing, with new capabilities can capture unique data about the user and his/her surroundings. As information capture becomes pervasive and subtle, people are naturally concerned about privacy and security~\cite{palen2003unpacking}. This is because the internet is ubiquitous but, have characteristics of a private space~\cite{camp2000internet}. Privacy in ubiquitous systems can be achieved by  identifying subtle trust assumptions of systems for design~\cite{camp2003designing} and adhering to privacy principles~\cite{langheinrich2001privacy}. An example of research which aims to uphold such principles include work by Egelman \etal , which evaluates various indicator designs to better communicate data capture on ubiquitous computing platforms. 

\subsection{Smartphones and Wearables Concerns}
Only efforts to improve end-user security which are grounded in user research directly address end-user concerns.The current state of smartphone security and privacy can be attributed to studies conducted to understand user attitudes toward security and privacy for smartphones~\cite{chin2012measuring}, observe behaviors of new smartphone users~\cite{palen2000going}, investigate user understanding of permissions~\cite{felt2012android}. 

Denning et. al conducted a study investigating privacy perspectives the bystanders of wearable devices ~\cite{Denning2014}. Bystanders expressed concerns due to the subtle, easy-to-use, and potentially ubiquitous nature of wearables. While this research examined the privacy concerns of bystanders in the presence of wearable devices, we are aware of no work that has looked at the privacy concerns of the owners of the devices.

\subsection{User Perceptions and Behaviors}
While user perceptions are insightful, it has been shown that people generally lack enough information to make privacy-sensitive decisions; even if they do, they are likely to trade off long-term privacy for short-term benefits ~\cite{acquisti2005privacy}. Actual behavior deviates from self-reported behaviors ~\cite{jensen2005privacy} and privacy preferences ~\cite{spiekermann2001privacy}. 

%The survey was constructed in a way to randomize the order of the particular sets of questions participants saw, except for the open-ended question, which was always near the end of the survey, asked along with the demographics. For this reason, people were heavily primed for the open-ended question. However, this question was always shown before the IUIPC questions, so our results on privacy being the top concern isn't because of the bias from the privacy index. The intent of the open-ended question  was more to get a sense of what people were concerned of, and we believe the results do reflect their actual concerns, but with a bit more clarity, since the participants were already thinking about such risks related to wearables. 

%%%%%%%%%%%%%%%%%%%

\section{Conclusion}

We surveyed 2,250 internet users to determine what contributes to a violation of privacy or security, which technologies are risky, and what users think are the biggest risk for operating wearable devices. Participants rated how upset they would be if 304 situations occurred, assessed the risk and benefit for 20 new technologies, and gave open-ended responses to express their concerns. We provide insight into how much and why data, recipient, and device contribute to users' perception of a situation, calibrate answers with existing smartphone literature, and provide a regression model. An assessment of a range of new technologies shows that users perceive new technologies to be low-risk and low-benefit, but we suspect this is due to limited exposure that an average person has with wearables techology. We also state what users perceived as the most significant concerns with respect to wearable devices. We conclude by discussing future research directions in the wearables and user study space. 

%NOTE: RESEARCH ETHICS/ ACKS COMMENTED OUT FOR ANONYMITY

%\section{Research Ethics} 
%We received advance approval from University of California, Berkeley's Institutional Review Board to perform this user study. Survey data was collected in an anonymous manner. Although the focus group was not conducted in an anonymous manner, participants were not asked to offer any confidential or sensitive information. 

%\section{Acknowledgments}
%The authors would like to thank The National Science Foundation's Graduate Research Fellowships Program (NSF GRFP), and the Intel Science and Technology Center for Secure Computing (also known as  Secure Computing Research for Users' Benefit, SCRUB), for the support of this research. Inspiration for this work emerged out of fruitful discussions at the Berkeley Laboratory for Usable and Experimental Security (BLUES) meetings. Many thanks go out to the many generous colleagues who have provided feedback and encouragement.


%%%%%%%%%%%%%%%%%%%
%%%            BIB & ACKS          %%%
%%%%%%%%%%%%%%%%%%%

\bibliographystyle{abbrv}
\bibliography{wearables_survey}  % wearables_survey.bib is the name of the Bibliography


%APPENDICES are optional
%\balancecolumns
\appendix
%Appendix A
\section{Fischhoff Prompts}
\label{sec:prompt} 

\textit{We would like to ask you to rate the <risks/benefits> associated with each of the following technologies.}

{\bf Risks:} \textit{Consider all types of risks: the risk of physical harm or death, the risk to others or bystanders, the financial cost of the technology, any distress caused by the technology, what the consequences would be if the technology was misused, any impact on the public, work, or personal life, and other considerations. (e.g. for electricity, consider the risk of electrocution, the pollution caused by coal, the risk that miners need to take to mine the coal, the cost to build the infrastructure to deliver electricity, etc.) Give a global estimate over a long period of time (say, a year) of both intangible and tangible risks.} \\[-.6cm]

\textit{Do not consider the costs or risks associated with these items. It is true, for example, that sometimes swimmers can drown. But evaluating such risks is not your present job. Your job is to assess the gross benefits, not the net benefits which remain after the costs and risks are subtracted out.} \\[-.6cm]

\textit{Please rate the following technologies below with a number. We know that this might be a bit hard to do, but please try to be as accurate as possible, adjusting the numbers until they feel they are right for you. Start with the least risky technology at 10 and assign higher numbers for the more risky technologies. (For instance, a technology rated 14 is half as risky as a technology rated 28.)}

{\bf Benefits:} \textit{Consider all types of benefits: how many jobs are created, how much money is generated directly or indirectly, how much enjoyment is brought to people, how much a contribution is made to the people's health and welfare, what this technology promotes, and so on. (e.g. for swimming, consider the manufacture and sale of swimsuits, the time spent exercising, the social interactions during swimming, and the sport created around the activity.) Give a global estimate over a long period of time (say, a year) of both intangible and tangible benefits.} \\[-.6cm]

\textit{Do not consider the costs or benefits associated with these items. It is true, for example, that electricity also creates a market for home appliances. But evaluating such benefits is not your present job. Your job is to assess the gross risks, not the net risks which remain after the costs and risks are subtracted out.} \\[-.6cm]

\textit{Please rate the following technologies below with a number. We know that this might be a bit hard to do, but please try to be as accurate as possible, adjusting the numbers until they feel they are right for you. Start with the least beneficial technology at 10 and assign higher numbers for the more beneficial technologies. (For instance, a technology rated 34 is twice as beneficial as a technology rated 17.)}




%\section{Risk and Benefit Rankings}
%\label{sec:techrank}
%
%BENEFIT: 
%
%technology 	median	25\%		75\% \\
%electricity	87	50	100\\
%internet	65	45	100\\
%laptops	60	40	80\\
%email	50	30	75\\
%smartphones	50	32.5	75\\
%location tracking	40	20	67.5\\
%heart rate detection	40	28	62.5\\
%language detection	35	15	60\\
%voice recognition	25	15	40\\
%speech to text	25	15	36.25\\
%lawnmowers	24	15	40\\
%facial recognition	22	13	40\\
%guns	20	10	30\\
%motrocycles	20	12	40\\
%voice based emotion detection	20	10	30\\
%facial detection	20	10	32\\
%discrete camera	20	15	30\\
%smartwatches	20	10	32.5\\
%google glass	20	12	40\\
%fitness trackers	19	10	30\\
%cubetastic	18	10	30\\
%discrete microphone	15	10	20\\
%age detection	12	10	21\\
%gender detection	10	10	15
%
%RISK: 
%
%technology 	median	25\%		75\%\\
%guns	60	40	80\\
%motrocycles	45	27	70\\
%electricity	25	15	40\\
%lawnmowers	20	12	30\\
%facial recognition	17	12.5	30\\
%internet	15	10	30\\
%discrete camera	12	10	30\\
%location tracking	10	10	20\\
%age detection	10	10	15\\
%gender detection	10	10	12\\
%language detection	10	10	10\\
%voice based emotion detection	10	10	15\\
%facial detection	10	10	25\\
%voice recognition	10	10	15\\
%heart rate detection	10	10	10\\
%email	10	10	16\\
%speech to text	10	10	10\\
%discrete microphone	10	10	20\\
%smartphones	10	10	19\\
%laptops	10	10	15\\
%fitness trackers	10	10	10\\
%smartwatches	10	10	10\\
%google glass	10	10	20\\
%cubetastic	10	10	30

%\section{Coding Label Definitions}
%\label{sec:coding}
%
%Privacy: ``privacy,'' revealing personal information, spying. \\
%Security:  ``security,'' compromise, malware, hacking. \\
%GPS tracking: ``location,'' ``GPS,'' being monitored. 
%
%Unaware use: using data without permission or in a different way than understood by user. \\
%Unaware collection: collecting data without permission. \\
%Unaware access: disclosure of data without permission. \\
%False information: inaccurate or maliciously false data.
%
%Health Risk: radiation, cancer, or long-term effects.\\
%Safety: distractions causing car crashes or injuries, mugging or violence because of the device, injuries from device malfunctions (battery burns).\\
%Discomfort: eye strain, headache, obscured vision, irritation. \\
%Financial cost: getting ripped off by buying the device or device accessories, having to buy another device when broken or stolen, financial compromise caused by device. \\
%Theft: the device getting stolen. 
%
%Social Impact: dependency, distance from friends and family, changes in decision making, social changes, etc. \\
%Social Stigma: judgment, hate, or bystander discomfort.\\ 
%Aesthetics: fashion, the device being ugly, mentions of not looking cool/dorky. 
%
%Miscellaneous: odd comments, uncommon concerns. \\
%None: ``None,'' no threat, perceiving no big concerns \\
%Don't know: ``do not know,'' hinting at confusion \\
%Don't care: `` do not care,'' nonchalant answers 


%\section{Detailed Tech Rankings}

\balancecolumns

% That's all folks!
\end{document}
\\
