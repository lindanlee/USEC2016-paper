\section{Related Work}
We discuss related work that has examined users' perceptions surrounding security and privacy risks.

\subsection{Ubiquitous Sensing}
Many authors have emphasized that we are rapidly moving towards a world of ubiquitous sensing and data capture, with ensuing privacy challenges~\cite{abowd2000charting,palen2003unpacking,camp2000internet}. Many researchers have worked to study how privacy can be preserved in such a future. Examples of such efforts include frameworks to design for privacy~\cite{bellotti1993design,camp2003designing,langheinrich2001privacy} or evaluate privacy~\cite{scholtz2004toward} in ubiquitous computing applications. Others have suggested various models for understanding privacy in ubiquitous computing systems~\cite{hong2004privacy, jiang2002approximate}.
However, none of these works attempted to quantify or rank user concern over different privacy risks.

\subsection{Smartphones and Wearables Concerns}
Many researchers have attempted to study end-user concerns about security or privacy issues associated with their smartphones~\cite{chin2012measuring, palen2000going, felt2012android}. 

Recently Denning et. al studied privacy concerns bystanders have when others around them are wearing a wearable device~\cite{Denning2014}. Participants in their study expressed concerns due to the nature of wearables (subtle UI and potenitally ubiquitous). While their research examined the privacy concerns of bystanders in the presence of wearables, we are aware of no work that has looked at the privacy concerns of owners.

\subsection{User Perceptions and Behaviors}
In this paper we focus on measuring people's perceptions of security and privacy risks.
One limitation of user perceptions is that people don't always have enough information to make privacy-sensitive decisions; even if they do, they often trade off long-term privacy for short-term benefits~\cite{acquisti2005privacy}. Also, actual behavior may deviate from self-reported behaviors~\cite{jensen2005privacy} and privacy preferences~\cite{spiekermann2001privacy}. 