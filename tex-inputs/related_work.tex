%!TEX root = ../paper.tex

\section{Related Work}
%%% better introduction, or take this out. This seems awkward. 
%We discuss related work that has examined users' perceptions surrounding security and privacy risks.

\subsection{Ubiquitous and Wearable Devices}
Many authors have emphasized that we are rapidly moving towards a world of ubiquitous sensing and data capture, with ensuing privacy challenges~\cite{abowd2000charting,palen2003unpacking,camp2000internet}. Many researchers have worked to study how privacy can be preserved in the presence ubiquitous devices. Examples of such efforts include frameworks to design for privacy~\cite{bellotti1993design,camp2003designing,langheinrich2001privacy}, protocols for anonymous communication {cornelius2008anonysense}, evaluation metrics for privacy ~\cite{scholtz2004toward}, and privacy models ~\cite{hong2004privacy, jiang2002approximate}. However, none of these works attempted to quantify or rank user concern over different privacy risks. 

Roesner et al.\ urge the community to address potential concerns for wearable devices before the technologies become widespread \cite{roesner2014security} and explore the unique and difficult problems these devices present in terms for law and policy \cite.{roesner2014augmented}. A small-scale interview of how bystanders feel about wearable devices found that bystanders were predominantly split between having indifferent and negative reactions to the device but expressed clear interest in being able to give permissions for the data being captured \cite{denning2014situ}. We hope our research furthers   work in communicating with users and bystanders. 

\subsection{Smartphones}
Many researchers have attempted to study user concerns about security or privacy issues associated with their smartphones. Research shows that perceptions of risk change as computing systems change. Chin et. al\ investigate how user attitudes toward security and privacy for smarphones differ from attitudes for traditional computing systems, finding that these attitudes differ significantly ~\cite{chin2012measuring}. How people use these smartphones was also unlike other computing systems. Palen et. al\ track smartphone users for six weeks to find rapid modifications in behaviors with respect to context and the social appropriateness of phone use \cite{palen2000going}. People also perceive and behave without a realistic understanding of the risks they are taking. Felt et. al\ examine the Android permission system and find that 17\% of participants paid attention to permissions, and only 3\% comprehended what these permissions meant.  

Smartphones enabled new systems which used new types of data which were previously hard to measure. Users tend to use these applications for their benefit, but do not think of the privacy implications. Lindqvist et. al\  studies a popular location tracking application, and find that people share their location information for gaming, signaling availability to friends, and more withou concern for privacy implications of broadcasting this information \cite{lindqvist2011m}. Tsai et. al\ found that mobile users do get feedback about releasing data, such as who has viewed that you were at a certain location, greatly impacts user behaviors \cite{Tsai2009}. However, this type of feedback is not generally provided to users. There are still many unresolved concerns, such as the opaqueness which prevents knowledge of what applications are doing with given data, along with rogue applications which steal your data \cite{1_kane_2010, zhou2011taming}.


\subsection{User Perceptions and Behaviors}
In this paper we focus on measuring people's perceptions of security and privacy risks. One limitation of user perceptions is that people don't always have enough information to make privacy-sensitive decisions; even if they do, they often trade off long-term privacy for short-term benefits~\cite{acquisti2005privacy}. Also, actual behavior may deviate from self-reported behaviors~\cite{jensen2005privacy} and privacy preferences~\cite{spiekermann2001privacy}.

While risk communication for the physical world has been examined for several decades (e.g.,~\cite{Fischhoff,Morgan2001}), research into effectively communicating computer-based risks has only recently been researched. For example, both Garg et al.\ and Blythe et al.\ show that due to varying perceptions and abilities that correlate with demographic factors, computer-based risk communication should employ some degree of demographic targeting~\cite{Garg2012,Blythe2011}. While this work is likely applicable to wearable computing risk communication, we believe that a better understanding of users' risk perceptions in this domain is warranted prior to examining risk communication.