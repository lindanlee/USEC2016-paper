%!TEX root = ../paper.tex

\section{Additional Related Work}
%%% better introduction, or take this out. This seems awkward. 
%We discuss related work that has examined users' perceptions surrounding security and privacy risks.
%In this section, we outline previous research examining privacy in wearable computing environments, smartphone privacy and security, as well as user risk perceptions and behaviors.

\subsection{Wearables and Privacy}
We are rapidly moving towards a world of ubiquitous sensing and data capture, with ensuing privacy challenges~\cite{abowd2000charting,palen2003unpacking,camp2000internet}. Roesner {\it et al.}\ urge the community to address potential concerns for wearable devices before the technologies become widespread~\cite{roesner2014security} and explore the unique and difficult problems these devices present in terms for law and policy~\cite{roesner2014augmented}.

Many researchers have worked to study how privacy can be preserved in the presence of ubiquitous devices. Examples of such efforts include frameworks to design for privacy~\cite{bellotti1993design,camp2003designing,langheinrich2001privacy}, protocols for anonymous communication~\cite{cornelius2008anonysense}, evaluation metrics for privacy~\cite{scholtz2004toward}, and privacy models~\cite{hong2004privacy, jiang2002approximate}. Our work aims to augment works like these with an understanding of what privacy means to the end user. 

\subsection{Lessons form Smartphones}
Research shows that perceptions of risk change based on the particular device being used. Chin {\it et al.}\ found that users' attitudes toward security and privacy for smarphones significantly differed from attitudes towards traditional computing systems, stemming from differences in how people used these systems ~\cite{chin2012measuring}. Undoubtedly, this will hold true for wearables as well as wearables have a sparse, world-driven interaction model. 

%% LL: None of this seems that relevant! :( I was just citing things to cite them.. find better things to say. 

%Palen {\it et al.}\ tracked smartphone users for six weeks and observed that people also perceive and behave without a realistic understanding of the risks they are taking ~\cite{palen2000going}; Felt {\it et al.}\ examined the Android permission system and found that 17\% of participants paid attention to permission requests, and only 3\% comprehended what these permissions meant. For these reasons, we need to 

%Smartphones allow applications to access new types of data. While this tends to benefit users, they do not think of the privacy implications. Lindqvist {\it et al.}\  studied a popular location tracking application, and found that people share their location information for gaming, signaling availability to friends, without concern for the privacy implications of broadcasting that information~\cite{lindqvist2011m}.

%There are still many unresolved concerns such as the opaqueness that prevents users from fully understanding how applications are using their data or rogue applications inappropriately accessing data~\cite{1_kane_2010, zhou2011taming}.

Tsai {\it et al.}\ found that when mobile users get feedback about releasing data, such as who has viewed location information, it greatly impacts future behaviors~\cite{Tsai2009}. Although this type of feedback is not provided to smartphone users, there is potential for impacting wearables users and shaping their behaviors so that they keep users safe. 


\subsection{User Perceptions}
While risk communication for the physical world has been examined for several decades (e.g.,~\cite{Fischhoff,Morgan2001}), research into effectively communicating computer-based risks has only recently been researched. For example, both Garg {\it et al.}\ and Blythe {\it et al.}\ show that due to varying perceptions and abilities that correlate with demographic factors, computer-based risk communication should employ some degree of demographic targeting~\cite{Garg2012,Blythe2011}. While this work is likely applicable to wearable computing risk communication, we believe that a better understanding of users' risk perceptions in this domain is warranted, prior to examining risk communication.

One limitation of user perceptions is that people do not always have enough information to make privacy-sensitive decisions. Even if users did have this information, it has been shown that users often trade off long-term privacy for short-term benefits~\cite{acquisti2005privacy}. Furthermore, actual behavior may deviate from stated privacy preferences~\cite{spiekermann2001privacy}. However, understanding user concerns is a necessary first step not only for risk communication, but preventative measures in general against breaches of privacy and security in a new threat landscape. 