%!TEX root = ../paper.tex

\section{Related Work}
%%% better introduction, or take this out. This seems awkward. 
%We discuss related work that has examined users' perceptions surrounding security and privacy risks.

%%% turn these into paragraphs rather than subsections? Or maybe this will be a bad idea if I have too many sources. But try turning them into paragaphs.
\subsection{Wearables Concerns}
A small-scale interview of how bystanders feel about wearable devices \cite{denning2014situ} found that bystanders were predominantly split between having indifferent and negative reactions to the device. A variety of factors that make recording more or less acceptable, including what they are doing when the recording is being taken.  Additionally, bystanders are expressed interest in being able to give permissions for the data being captured. We also investigate how the type and mechanism of data capture affects privacy concerns, but we examine the privacy concerns of wearables owners at a large-scale while their research examined the privacy concerns of wearables bystanders at a small-scale. 

\subsection{Ubiquitous Sensing Concerns}
Many authors have emphasized that we are rapidly moving towards a world of ubiquitous sensing and data capture, with ensuing privacy challenges~\cite{abowd2000charting,palen2003unpacking,camp2000internet}. Many researchers have worked to study how privacy can be preserved in such a future. Examples of such efforts include frameworks to design for privacy~\cite{bellotti1993design,camp2003designing,langheinrich2001privacy} or evaluate privacy~\cite{scholtz2004toward} in ubiquitous computing applications. Others have suggested various models for understanding privacy in ubiquitous computing systems~\cite{hong2004privacy, jiang2002approximate}. However, none of these works attempted to quantify or rank user concern over different privacy risks. 
%one-sentence on takeaways from these sources and comment on them!

\subsection{Smartphones Concerns}
Many researchers have attempted to study end-user concerns about security or privacy issues associated with their smartphones~\cite{chin2012measuring, palen2000going, felt2012android}.  
%one-sentence on takeaways from these sources and comment on them!

\subsection{User Perceptions and Behaviors}
In this paper we focus on measuring people's perceptions of security and privacy risks. One limitation of user perceptions is that people don't always have enough information to make privacy-sensitive decisions; even if they do, they often trade off long-term privacy for short-term benefits~\cite{acquisti2005privacy}. Also, actual behavior may deviate from self-reported behaviors~\cite{jensen2005privacy} and privacy preferences~\cite{spiekermann2001privacy}.
%one-sentence on takeaways from these sources and comment on them!

While risk communication for the physical world has been examined for several decades (e.g.,~\cite{Fischhoff,Morgan2001}), research into effectively communicating computer-based risks is relatively nascent. For example, both Garg et al.\ and Blythe et al.\ show that due to varying perceptions and abilities that correlate with demographic factors, computer-based risk communication should employ some degree of demographic targeting~\cite{Garg2012,Blythe2011}. While this work is likely applicable to wearable computing risk communication, we believe that a better understanding of users' risk perceptions in this domain is warranted prior to examining risk communication.