\subsubsection{Data Recipient}
Across all scenarios, 42.3\% of participants stated that they would be ``very upset'' if their data was shared with only the app's servers, whereas the VURs for friends (69.5\%), work contacts (75.2\%), and the public (72.4\%) were much higher. A chi-square test indicated that these differences were statistically significant (Table \ref{recipient}). However, these effect sizes were small: the largest effect was between work contacts and an app's server ($\phi=0.11$); while the VUR for sharing with work contacts was significantly higher than sharing with friends, the effect size was negligible ($\phi=0.004$). 

We note that this chi-square test violates the assumption of independent observations, since participants responded to multiple scenarios. But based on the randomization of treatments and large sample size, we do not believe that this significantly impacted our results. Nonetheless, we repeated the analysis using only one randomly-selected data point per participant to find that the test was robust to this violation. Participants were significantly more concerned about having their data seen by humans ({\it vis-{\`a}-vis} app servers), though differences between specific human groups (between the public, friends, and work contacts) were not significant.

{\color {red} we're not saying that there is no distinction, just that there is more distinction between people and the app server."users make little distinction between sharing data with friends, co-workers, or the general public”: at first sight, this result is in conflict with privacy studies specifically in the social arena (e.g., Facebook). The authors may want to comment on this fact if they have any explanation.}