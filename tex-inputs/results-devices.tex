%!TEX root = ../paper.tex


\begin{table}[t]
\begin{center}
\begin{tabular}{| c | r | r |}
\hline
 Misbehavior &  Cubetastic3000 & Smartphone \\
 \hline
 \hline
 All & 58.79\% & 46.67\%\\
 \hline
Vibration & 14.81\%  &  6.14\%\\
Bluetooth & 44.12\%  &  19.86\%\\
Unmuted Call & 87.10\%  &  58.44\%\\
Screenshot & 52.78\%  & 55.74\%\\
Premium Calls/Texts & 86.49\%  &  91.94\%\\ 
\hline
\end{tabular}
\caption{VURs for the five questions about device misbehaviors described in Section \ref{sec:smartphones}, contrasting smartphones with the Cubetastic3000.}
\label{deviceVUR}
\end{center}
\end{table}

\subsubsection{Device}
Recall that each participant answered 2 questions drawn from a set of 5 regarding their reactions to smartphone misbehaviors. To compare these misbehaviors with misbehaviors on the Cubetastic3000, we included these same 5 questions amongst the pool of 293 Cubetastic3000 scenarios, only modifying the device type. In this manner, while all 1,782 participants received 2 smartphone questions, there were 159 participants who received at least one of these questions in relation to the Cubetastic 3000. Across all participants, the VUR was 46.7\% (of 1,782) when describing smartphones, whereas the VUR increased to 58.8\% (of 159) when describing these same misbehaviors on the Cubetastic3000. The VURs for both devices for all 5 questions are in Table ~\ref{deviceVUR}.

To ensure independence of observations, we performed a Mann-Whitney U test by comparing participants' average VURs for the Cuebtastic3000 scenarios (i.e., 159 participants) to the remaining participants' average VURs for the smartphone scenarios (i.e., 1,623 participants). We found this difference to be statistically significant ($108,664.0$, $p<0.0005$), however, the effect size was incredibly small ($r=0.08$). Because of this small effect size, we did not further reduce our statistical power by separately comparing each of the 5 misbehaviors. As a result, we can conclude that in general users are likely to be more wary of misbehaviors occurring on wearable devices than smartphones, the difference is likely negligible. Similarly, the entire effect may be due to participants' increased familiarity with smartphones, and therefore may disappear as they increasingly encounter more wearable devices.

%{\color {red} talk about the variance of the devices in general--was there more spread for the answers wrt smartphones, cubetastic, or were they kind of all the same?}



						
%\begin{table}%[h]
%\begin{center}
%\begin{tabular}{| c | c | c | c |}
%Question & $\chi^2$ &	2-tail P & Effect Size \\
%All & 0.444  & 0.505 & 0.006\\
%Q1 & 0 & 1 & 0 \\
%Q2 & 0 & 1 & 0\\
%Q3 & 0 & 1 & 0\\
%Q4 & 0 & 1 & 0\\
%Q5 & 0 & 1 & 0\\
%\end{tabular}
%\caption{Results of the effects of <device> type contributing to VUR. Values are McNeamar's test results for within-subjects comparisons for participants who received both questions.}
%\label{withindevice}
%\end{center}
%\end{table}

%For the between-subjects comparison (i.e., participants who received one question, but not the other), you can do either a chi-square test or Fisher's exact test. Both use a 2 x 2 contingency matrix (i.e., rows are outcomes---upset or not---and columns are conditions---wearable or smartphone). The way to choose between the two tests is based on sample size, generally you use chi-square when there are more than 10-20 samples per cell in the table, but using either is just as valid. For the within-subjects comparisons (i.e., participants received both questions), you would do McNemar's test; which is the within-subjects version of the chi-square test. 
