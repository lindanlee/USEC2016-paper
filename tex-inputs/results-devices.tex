%!TEX root = ../paper.tex

\subsubsection{Device}
{\color {red} intro to this section} 

The five questions were:
\begin{enumerate}[topsep=0pt,itemsep=-1ex,partopsep=1ex,parsep=1ex]
\item Q1: How would you feel if an app on your <device> vibrated your phone without asking you first?
\item Q2: How would you feel if an app on your <device> connected to a Bluetooth device (like a headset) without asking you first?
\item Q3: How would you feel if an app on your <device> un-muted a phone call without asking you first?
\item Q4: How would you feel if an app on your <device> took screenshots when you were using other apps, without asking you first?
\item Q5: How would you feel if an app on your <device> sent premium (they cost money) calls or text messages, without asking you first?
\end{enumerate}

Participants had unique VURs for scenarios only differing in device. Our participants had a 58.79\% VUR when asked about wearables and 46.64\% VUR when asked about smartphones. The VURs for both devices for all 5 questions are in table ~\ref{deviceVUR}. {\color {red} talk about the standard deviation of the devices in general--was there more spread for the answers wrt smartphones, cubetastic, or were they kind of all the same?} However, the effect the device has on the VUR is not considered to be statistically significant (see Table ~\ref{betweendevice}). Additionally, there is no statistically significant difference between how people reacted in a given situation; although, participants were statistically significantly upset in Q2. The aforementioned results are only with respect to between subjects analysis, where answers are from participants who received either only the wearables or smartphone version of the 5 questions. Too few instances of participants answering both versions of questions occurred (34 in total for all 5 questions) to perform a sound within-subjects analysis.  

\begin{table}[t]
\begin{center}
\begin{tabular}{| c | r | r |}
\hline
 Question &  Wearable VUR, $\sigma$ & Smartphone VUR, $\sigma$ \\
 \hline
 All & 58.79\%, \# & 46.64\%, \# \\
Q1 & 14.81\%,  \# &  6.13\%, \# \\
Q2 & 44.11\%, \# &  19.85\%, \# \\
Q3 & 87.09\%, \# &  58.44\%, \# \\
Q4 & 52.77\%, \# & 55.74\%, \# \\
Q5 & 86.49\%, \# &  91.82\%, \# \\ 
\hline
\end{tabular}
\caption{VURs for the questions described in Section \ref{sec:smartphones}, contrasting smartphones with the Cubetastic3000.}
\label{deviceVUR}
\end{center}
\end{table}

\begin{table}%[h]
\begin{center}
\begin{tabular}{|c|r|r|r|r|}
%Question & Chi^2 &	2-tail P &	Sig?	 & n	& effect size\\
%All	2.814	0.1395	No		3589		0.001\\
%Q1	2.500	0.1139	No		714		0.004\\
%Q2	17.333	<0.0001	Yes		708		0.024\\
%Q3	0.020	0.8886	No		699		0.000\\
%Q4	1.426	0.2324	No		731		0.002\\
%Q5	1.611		0.2043	No		710		0.002\\
\hline
Question & $\chi^2$ & p-value & n & $\phi$ \\
\hline
All & 2.202 & <0.1378 & 3,588 & 0.001\\
Q1 & 2.500 & <0.1139 & 714 & 0.004\\
Q2 & 17.333 & <0.0001 & 708 &  0.024\\
Q3 & 0.020 & <0.8886 & 699 &  0.000\\
Q4 & 1.413 & <0.2345 & 730& 0.002\\
Q5 & 1.604 & <0.2054 & 709 & 0.002\\
\hline
\end{tabular}
\caption{Chi-square test results comparing participants' VURs between the smartphone and Cubetastic3000 questions.}
\label{betweendevice}
\end{center}
\end{table}
						
%\begin{table}%[h]
%\begin{center}
%\begin{tabular}{| c | c | c | c |}
%Question & $\chi^2$ &	2-tail P & Effect Size \\
%All & 0.444  & 0.505 & 0.006\\
%Q1 & 0 & 1 & 0 \\
%Q2 & 0 & 1 & 0\\
%Q3 & 0 & 1 & 0\\
%Q4 & 0 & 1 & 0\\
%Q5 & 0 & 1 & 0\\
%\end{tabular}
%\caption{Results of the effects of <device> type contributing to VUR. Values are McNeamar's test results for within-subjects comparisons for participants who received both questions.}
%\label{withindevice}
%\end{center}
%\end{table}

%For the between-subjects comparison (i.e., participants who received one question, but not the other), you can do either a chi-square test or Fisher's exact test. Both use a 2 x 2 contingency matrix (i.e., rows are outcomes---upset or not---and columns are conditions---wearable or smartphone). The way to choose between the two tests is based on sample size, generally you use chi-square when there are more than 10-20 samples per cell in the table, but using either is just as valid. For the within-subjects comparisons (i.e., participants received both questions), you would do McNemar's test; which is the within-subjects version of the chi-square test. 