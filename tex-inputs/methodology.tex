%!TEX root = ../paper.tex

\section{Methodology}
To obtain a comprehensive list of possible risks that wearable devices might present in the future, we examined the sensors, capabilities, permissions, and applications of the most popular wearable devices on the market. At the time of this study (August 2014) the most popular wearable devices included the Fitbit fitness tracker, which continuously monitors heartbeat, steps taken, and sleep patterns;
%~\cite{6_fitbit_2014, 7_time_2014}
the Pebble smartwatch, which can take pictures, send texts, show notifications from online, and push notifications to services; %~\cite{pebble_smartwatch_2014, 9_verge_2014, 10_readwrite_2014}
and Google Glass, which can take pictures, record video, and perform a subset Internet-based tasks such as search, reading emails, etc. %~\cite{11_wikipedia_2015, 12_turi_2014}. 
These devices' capabilities and requested permissions as inspiration to develop a list of possible security and privacy risks that users will encounter in use.

We designed a survey to gauge the relevancy of all these possible risks.
Our survey contained two main sections. A set of questions presented participants with several scenarios---something undesirable that might happen with their wearable device---and asked them to rate their level of concern if each scenario were to happen. This was intended to elicit their perception of the severity and impact of the risk.
The format of this section was based on Felt {\it et al.}'s study of user perceptions of security and privacy risks with mobile devices~\cite{Felt}. Another set asked participants to compare the risks and benefits of wearable technologies to those of better-understood technologies, following the same methodology from Fischhoff {\it et al.}'s seminal study in risk perception~\cite{Fischhoff}.
%Finally, we collected demographic information, which included a privacy concern scale and whether participants owned any wearable devices. 

\subsection{Related Work}
In this section, we describe the two prior works on which we based our survey format: Felt {\it et al.}'s survey of smartphone-based risks~\cite{Felt} and Fischhoff {\it et al.}'s survey of a wide range of general technology-based risk perceptions~\cite{Fischhoff}.

\subsubsection{Smartphone Risk Scenarios}
Felt \etal previously studied the security concerns of smartphone users by conducting a large-scale online survey~\cite{Felt}. Their survey asked 3,115 smartphone users about 99 risk scenarios. Participants were asked how upset they would be if a certain action occurred without their permission. Participants rated each situation on a Likert scale ranging from ``indifferent (1)'' to ``very upset (5).''
Our methodology closely follows that study, but with scenarios chosen to shed light on the security and privacy risks of wearable devices.

\subsubsection{Technology Risk Perceptions}
Fischhoff \etal performed a seminal study of the perceived risks and benefits surrounding 30 widely used technologies~\cite{Fischhoff}. Participants were asked to separately think of the risks then benefits, considering all people affected, long-term effects, and short term effects. Then, the participants respectively rated these technologies on a numerical scale, being instructed to rate the least risky or least beneficial technology a 10 and scaling the ratings linearly (e.g., a technology with a risk rating of 20 is considered twice as risky as compared to a technology with a risk rating of 10). We apply their methodology to evaluate perceived risks and benefits of technologies related to wearable computing with respect to more familiar technologies.
%Using this methodlogy, they were able to categorize different technologies based on whether they were high-risk/high-benefit, low-risk/low-benefit, and so forth. 

\subsection{Survey Questions}
\noindent Each participant answered 27 questions across 5 sections:   \\[-.5cm]

\begin{itemize} \itemsep1pt \parskip0pt \parsep0pt
\item 2 reading comprehension questions
\item 6 questions about wearable computing scenarios 
\item 2 questions about smartphone scenarios 
\item 2 Fischhoff-style risk/benefit questions 
\item 15 demographic questions %\\[-.8cm]
\end{itemize}

We randomized the order participants saw sections of the survey (with the exception of the comprehension and demographic questions, which were always first and last, respectively), as well as the order of questions in each section.

\subsubsection{Comprehension Questions}
Because participants might be biased to specific companies (e.g., visceral reactions to Google Glass based on popular media stories), we based our questions on a fictitious wearable. Thus, the beginning of the survey introduced participants to the ``Cubetastic3000,'' which was the basis for all questions on wearables risks. We highlighted the capabilities of this device and described use cases:

\begin{quotation}
{\it Imagine that you are the proud owner of the Cubetastic3000, a new, high-tech computing device designed to be worn on your head. Imagine also that you wear this device all the time, because it is very lightweight, durable, and convenient.

The Cubetastic3000 has the capability to capture video, photos, audio, and biometrics (biological data about you, such as heart rate). Just like other devices today, you can install third-party applications from an app store, and these applications can use the information that the Cubetastic3000 captures.

With a wide range of applications and capabilities, your device can do all sorts of things, such as:\\

\noindent---measuring heart rate, breathing, and other things to keep track of your fitness level and overall health\\

\noindent---look at what you see to provide information about what's around you\\

\noindent---allow you to take notes just by telling the device what you need to remember\\

\noindent---take videos of you or what you see to share with others\\

\noindent---automatically take photos or video so that you can replay events that previously happened\\

\noindent---play music that you like for you when it detects that no one is around\\

\noindent---infer information about you so you don't need to log in or search for the same thing over and over\\

\noindent ...and much more!}
\end{quotation}

Since this device is basis for many of our survey questions, we ensure that participants had understood this device's capabilities by asking them two multiple-choice comprehension questions about the device's. We filter out responses from participants who could not answer these questions.

\subsubsection{Wearable Scenarios}
We present scenarios involving data captured by the Cubetastic3000 and ask participants to rate how upset they would be if a particular type of data (e.g., video, audio, name, etc.) were shared without permission with a particular data recipient (see Figure \ref{fig:prompt}). Responses were reported on a 5-point Likert scale (from ``indifferent'' to ``very upset''), following Felt \etal\cite{Felt}. Questions were of the form: 

\begin{quotation}
\noindent
\textit{``How would you feel if an app on your Cubetastic3000 learned <data> and shared it with <recipient>, without asking you first?''}
\end{quotation}

We created an initial pool of 288 questions by combining 72 data types (<data>) with 4 data recipients (<recipient>): %\\[-.8cm]

\begin{figure}[t]
	\centering
	\includegraphics[width=\columnwidth]{images/prompt.pdf}
	\caption{An example of a wearable scenario question participants saw while taking the survey.}
	\label{fig:prompt}
\end{figure}

\begin{packed_item}
\item Your work contacts
\item Your friends
\item The public
\item The app's server (but didn't share it with anyone else) %\\[-.8cm]
\end{packed_item}

The purpose for using this question format was to determine how upset participants would be if data were inappropriately shared, and the extent to which their reactions were based on the data type and recipient. Each participant answered 6 questions that were randomly drawn from a pool of 293: the 288 described here, plus 5 that we describe in the next section (Section \ref{sec:smartphones}).

\subsubsection{Smartphone Scenarios}
\label{sec:smartphones}
We presented participants with a second set of scenarios to control for the type of device being used. Rather than using the previous pool of 288 <data> and <recipient> combinations, we selected 5 scenarios that Felt \etal found least and most concerning to their participants~\cite{Felt}:

\begin{quotation}
\begin{enumerate}[topsep=0pt,itemsep=-1ex,partopsep=1ex,parsep=1ex]
\item {\it How would you feel if an app on your <device> vibrated your phone without asking you first?
\item How would you feel if an app on your <device> connected to a Bluetooth device (like a headset) without asking you first?
\item How would you feel if an app on your <device> un-muted a phone call without asking you first?
\item How would you feel if an app on your <device> took screenshots when you were using other apps, without asking you first?
\item How would you feel if an app on your <device> sent premium (they cost money) calls or text messages, without asking you first?}
\end{enumerate}
\end{quotation}

In the previously described section of our survey, <device> was set to ``Cubetastic3000'' and not every participant received one of these questions (i.e., these 5 questions were among the pool of 293 questions from which participants were randomly assigned 6). In the separate smartphone section of the survey, every participant received exactly two of these questions, where <device> was set to ``smartphone.'' This allowed us to perform controlled comparisons based on whether the same misbehavior was occurring on a smartphone (i.e., a better understood device) or the Cubetastic3000 (i.e., a fictitious wearable device).

\subsubsection{Risk and Benefit Assessment}
In addition to investigating reactions to particular scenarios, we examine broad perceptions of new technologies and how those compared to perceptions of other understood technologies. We model this section after a seminal risk perception study by Fischhoff \etal\cite{Fischhoff}, in which participants ranked technologies by their relative risk and benefit to society. We ask participants to perform this exercise for 4 technologies previously examined by Fischhoff {\it et al.}: handguns, motorcycles, lawnmowers, and electricity.  These technologies are chosen to span varying levels of risks and benefits.

Our goal was to ask about familiar technologies such as the Internet, general and specific wearable artifacts, and a range of new capabilities. To do this, we ask participants to evaluate one of 20 technologies relevant to wearables along 4 previously studied technologies. The 20 technologies are: Internet, email, laptops, smartphones, smart watches, fitness trackers, Google Glass, Cubetastic3000, discrete camera, discrete microphone, facial recognition, facial detection, voice recognition, voice-based emotion detection, location tracking, speech-to-text, language detection, heart rate detection, age detection, and gender detection.

To parallel Fischhoff {\it et al.}'s risk perception study, we gave our participants a similar prompt to numerically express the perceived gross risk/gross benefit over a long period of time for all parties involved. We randomized whether they performed the ranking for risks or benefits first. The prompt is listed in Appendix \ref{sec:prompt}. The question format was as follows:

\begin{quotation}
{\it \noindent Fill in your <risk/benefit> numbers for the following:\\

\noindent Handguns: \_\_\_\_\_\_\_ \\
Motorcycles: \_\_\_\_\_\_\_\\
Lawnmowers: \_\_\_\_\_\_\_\\
<Wearable Technology>: \_\_\_\_\_\_\_\\
Electricity: \_\_\_\_\_\_\_\\ }
\end{quotation} 

\subsubsection{Additional Questions}
The exit portion of the survey contained demographic questions asking for age, gender, and education. We also asked participants if they owned a wearable device so we could control for prior exposure, and included an open-ended question on what would be the most likely risks associated with wearable devices. We ended with the 10-question Internet Users' Information Privacy Concerns (IUIPC) index~\cite{malhotra2004internet}, so we could control for participants' general privacy attitudes.

\subsection{Focus Group}
We conducted a one-hour focus group to validate our design, gauge comprehension, and measure fatigue. The focus group began with participants taking the survey. Afterward, we asked participants to give feedback on the format and the content, noting any instructions or questions that were unclear. The focus group concluded with a discussion of possible benefits and risks of wearable devices, in order to brainstorm any additional scenarios to include. Finally, we compensated participants with \$30 in cash. We recruited all of our focus group participants from Craigslist. Of the 13 participants, 54\% were female, and ages ranged from 18 to 64 ($\mu = 36.1$, $\sigma = 15.3$).  Education backgrounds ranged from high school to doctorate degrees, and professions included student, artist, marketer, and court psychologist.

\subsection{Recruitment and Analysis Method}
We recruited 2,250 participants over August 7--13, 2014 via Amazon's Mechanical Turk. We restricted participants to those over 18, living in the United States, and having a successful HIT completion rate of 95\% or above. We compensated each participant with \$1.75 upon successfully completing the survey. Based on incorrect responses to either of the two comprehension questions, we filtered out 366 (16\% of 2,250) participants. We filtered out an additional 99 participants (4\% of 2,250) due to incomplete responses, and three participants who were under 18, leaving us with a total sample size of 1,782. Of these, 57.9\% were male (1,031), 41.0\% were female (731), and 20 participants declined to state their genders. Ages ranged from 18 to 73, with a mean of 32.1 ($\sigma$ = 10.37). Almost half of our participants had completed a college degree or more (49.2\% of 1,782), which includes the 219 (12.3\% of 1,782) who reported graduate degrees. While our sample was younger and more educated than the U.S. population as a whole, we believe it is still consistent with the U.S. Internet-using population.

In performing our analysis in the next section, we chose to focus on the very upset rate (VUR) of each scenario.  The VUR is defined as the percentage of participants who reported a `5' on the Likert scales. 
We use the VURs rather than the average of all Likert scores for the same reasons as Felt {\it et al.}: the VUR does not presume that the ratings, ranging from ``indifferent'' to ``very upset,'' are linearly spaced. Additionally, most people are likely to be upset, at least a little, in all scenarios, because a device is taking action without permission (rating distribution: ``1''= 759, ``2'' = 918, ``3'' = 1,452, ``4''' = 2,421, ``5'' = 8,344). Thus, the main distinguishing factor of a participant reacting to a given scenario is whether they were maximally upset or not, rather than how upset they were. However, one limitation of this approach is that it only allows us to make {\it relative} comparisons between scenarios, rather than being able to definitively state how upset people might be if a single scenario were to occur.

We followed Fischhoff {\it et al.}'s methodology and did not normalize the numerical responses. Rather, we report medians and quartiles, which are not impacted by outliers. For the open-ended question at the end (i.e., additional privacy concerns), two researchers independently coded 1,782 responses, with an initial agreement rate of 89.7\%. The researchers discussed and resolved any disagreements so that the final codings reflect unanimous agreement.
