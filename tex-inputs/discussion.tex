%!TEX root = ../paper.tex

\section{Discussion}
Here, we discuss complementary future research directions in fields of privacy, ubiquitous computing, and user studies, along with specific limitations of this survey.

\subsection{Interpreting the Survey Results}
One of the main limitations of this work is that our participants might not have interest in or knowledge of wearables and their capabilities. 83\% of our participants reported that they do not own a wearable device. These participants may have underestimated or overestimated the risk perceived in various scenarios. People may be overreacting to recent events for scenarios \footnote{During our study, there were many stories covering injuries from exploding batteries (http://www.bloomberg.com/news/articles/2014-08-11/exploding-lithium-batteries-riskier-to-planes-research), which were explicitly and repeated mentioned when self-reporting concerns.}. People may be underestimating the risk of sharing certain data due to unawareness of what can be inferred from the data, or not have an idea of how to rate a new technology with respect to familiar ones. Biometrics were generally not a concern for our participants, although there are many security and privacy implications \cite{prabhakar2003biometric}. {\color {red} Any economics or behavioral papers to support our claims and elaborate on this would be great here. Maybe ones on perception, estimation, etc.}

We believed that getting a representative survey base was a useful endeavor. We could have easily recruited only wearables owners or people specifically interested in wearables. However, that will also have its own biases and limitations, since this does not reflect the general population. At the time of this writing, about 85\% of the general population do not own wearable devices \cite{Nilsen,WearableStatNews}, so our study reflective of the status quo. We expect user perceptions to change as rapidly as wearable technologies and the rate of adoption change. 

Privacy concerns asked out of context differ from how users may react to these same concerns in real life \cite{norberg2007privacy, jensen2005privacy}. This is an unavoidable, yet important consideration of any study of this nature. This privacy paradox means that our findings may not be exactly representative of how upset users may be in real life, but do reflect their perceptions of wearable devices and various associated scenarios. 

\subsection{Future Research Directions}
Further work can be done to expand various aspects of this study. Investigating more fine-grained data types (e.g. investigating if various types of location data, versus just location data in general) would be a useful endeavor to gain further insight into user perception. Adding more recipients, like ``advertisers'' or ``acquaintances'' may lead to more contrasting results.  

While privacy and security concerns were expected, consider the following self-reported user concerns as inspiration for future research: addressing the high financial costs of wearables, communicating the reality of health concerns from constant use, creating distraction-free interfaces to prevent safety issues, minimizing negative social impacts of wearable device use, and improving device aesthetics.  

Wearables are still in infancy. Perceptions of situations and capabilities will change rapidly with advancements and increased exposure. However, videos and textual information are considered to be significantly sensitive by our participants, along with past participants of smartphone user perception studies. Various systems which detect and take actions for sensitive objects in photos and videos will be critical as wearables and other devices become more ubiquitous.