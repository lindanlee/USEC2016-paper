%!TEX root = ../paper.tex

\section{Discussion}
%We discuss considerations when interpreting our results, the limitations of our work, and future work we hope to inspire. 

%\subsection{Limitations}
One of the limitations is that our participants might not have interest in or knowledge of wearables and respective capabilities. 83\% of our participants reported that they do not own a wearable. Participants may be over or underestimating the risk due to unawareness of what can be inferred from the data, not have an idea of how to rate a new technology with respect to familiar ones, or be greater influenced by recent events for scenarios.\footnote{Recently, stories  of  exploding batteries were in the news \cite{1_levin_2014}, which were explicitly self-reported as a concern.}.  For instance, biometrics were generally not a concern for our participants, although there are many security and privacy implications \cite{prabhakar2003biometric}.

We believed a representative survey base was a useful endeavor. We could have easily recruited only wearables owners or people specifically interested in wearables. However, that will have its own biases and limitations, since this does not reflect the general population. At the time of this writing, about 85\% of the general population do not own wearable devices \cite{Nilsen,WearableStatNews}, making our study is reflective of the current population. 

Privacy concerns asked out of context differ from how users may react to these same concerns in real life \cite{norberg2007privacy, jensen2005privacy}. This privacy paradox means that our findings may not be representative of how upset users may react to these scenarios in real life, but do reflect actual perceptions of wearable devices and various associated scenarios. This is an unavoidable, yet important distinction to make with of studies of our nature.

%\subsection{Future Research Directions}
Further work can be done to expand various aspects of this study. Investigating more fine-grained data types (e.g. investigating various types of location data, versus just location data in general) would be a useful endeavor to gain further insight into user perception. Adding more recipients, like ``advertisers'' or ``acquaintances'' may lead to more contrasting results. Additionally, the following self-reported user concerns as inspiration for future research: addressing the high financial costs of wearables, communicating the reality of health concerns from constant use, creating distraction-free interfaces to prevent safety issues, minimizing negative social impacts of wearable device use, and improving device aesthetics.  

Wearables are still in their infancy. Perceptions of situations and capabilities will change rapidly with advancements and increased exposure. However, videos and textual information are currently and significantly considered to be sensitive by our participants, along with past participants of smartphone user perception studies. Various systems which detect and take actions for sensitive objects in photos and videos will be critical as wearables and other devices become more ubiquitous.