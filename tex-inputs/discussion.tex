%!TEX root = ../paper.tex

\section{Discussion}
%We discuss considerations when interpreting our results, the limitations of our work, and future work we hope to inspire. 

\subsection{Limitations}
One of the limitations of our experiment is that our participants might not have knowledge or interest in wearables and their capabilities; 83\% of our participants reported that they do not own a wearable. Because of this, our participants may be over or underestimating the risk, stemming form an unawareness of what can be inferred from the data, not having clear relations of new technology with respect to familiar ones, and a higher likelihood of being influenced by reports of recent events.\footnote{Recently, stories of exploding batteries were in the news~\cite{1_levin_2014}, which were explicitly reported as a concern in our open-ended question.}  For instance, biometrics were generally not a concern for our participants, although there are many security and privacy implications~\cite{prabhakar2003biometric}. Our participants also did not differentiate between the benefits of risks of various new capabilities.

We recruited both wearable users and non-users in order to yield a more representative sample of the general population. We could have easily recruited only wearables owners or people specifically interested in wearables. However, that would have its own biases and limitations. At the time of this writing, about 85\% of the general population do not own wearable devices~\cite{Nilsen,WearableStatNews}, indicating our study is reflective of the current population. 

Because of the privacy paradox, participants' stated responses may differ from how they may react to these same scenarios in real life~\cite{norberg2007privacy, jensen2005privacy}. At the same time, our results do reflect actual perceptions of wearable devices and the associated privacy scenarios involving them. This is an unavoidable, yet important distinction to make with of studies of this nature: our primary goal was to examine perceptions and preferences, so that future systems can be designed with these in mind. We do not expect that such systems will satisfy users in all situations, however, we believe that user-centered design will still be a vast improvement over post hoc approaches (or ignoring user concerns altogether).

\subsection{Future Research Directions}
We find that although people have opinions on applications which are familiar, users do not know the risk associated with new data or unfamiliar applications. We hope our work both informs the direction for future research to secure video, audio, and other currently considered sensitive sensor input channels, but also encourage work for contextual and user-input- independent permission models and access control schemes.

Further work can be done to expand various aspects of this study. Investigating more fine-grained data types (e.g., investigating specific instances of location data, versus location data in general) would be a useful endeavor to gain further insight into user perceptions. Adding additional recipients, such as ``advertisers'' or ``acquaintances'' may lead to more nuanced results. Additionally, the open-ended concerns illuminate areas of possible future research, such as the design of a distraction-free interface to prevent safety issues, and how to minimize negative social impact. 