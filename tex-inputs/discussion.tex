%!TEX root = ../paper.tex

\section{Discussion}
%We discuss considerations when interpreting our results, the limitations of our work, and future work we hope to inspire. 

%\subsection{Limitations}
One of the limitations is that our participants might not have interest in or knowledge of wearables and their respective capabilities, since 83\% of our participants reported that they do not own a wearable. Participants may be over or underestimating the risk due to unawareness of what can be inferred from the data, not have an idea of how to rate a new technology with respect to familiar ones, or be more influenced by recent events.\footnote{Recently, stories of exploding batteries were in the news~\cite{1_levin_2014}, which were explicitly reported as a concern in our open-ended question.}  For instance, biometrics were generally not a concern for our participants, although there are many security and privacy implications~\cite{prabhakar2003biometric}.

We recruited both wearable users and non-users in order to yield a more representative sample of the general population. We could have easily recruited only wearables owners or people specifically interested in wearables. However, that would have its own biases and limitations. At the time of this writing, about 85\% of the general population do not own wearable devices~\cite{Nilsen,WearableStatNews}, indicating our study is reflective of the current population. 

Because of the privacy paradox, participants' stated responses may differ from how they may react to these same scenarios in real life~\cite{norberg2007privacy, jensen2005privacy}. At the same time, our results do reflect actual perceptions of wearable devices and the associated privacy scenarios involving them. This is an unavoidable, yet important distinction to make with of studies of this nature: our primary goal was to examine perceptions and preferences, so that future systems can be designed with these in mind. We do not expect that such systems will satisfy users in all situations, however, we believe that user-centered design will still be a vast improvement over post hoc approaches (or ignoring user concerns altogether).

%\subsection{Future Research Directions}
Further work can be done to expand various aspects of this study. Investigating more fine-grained data types (e.g., investigating specific instances of location data, versus location data in general) would be a useful endeavor to gain further insight into user perceptions. Adding additional recipients, such as ``advertisers'' or ``acquaintances'' may lead to more nuanced results. Additionally, the open-ended concerns are inspiration for future research: addressing the high financial costs of wearables, communicating the reality of health concerns from constant use, creating distraction-free interfaces to prevent safety issues, minimizing negative social impacts of wearable device use, and improving device aesthetics.  

Wearables are still in their infancy. Perceptions of situations and capabilities will change rapidly with advancements and increased exposure. However, pursuant to previous studies of smartphone risk perceptions, our participants found embarrassing videos and risks involving financial losses to be the most sensitive, which suggests that these risks should be highlighted by platform developers. Various systems which detect and take actions for sensitive objects in photos and videos will be critical as wearables and other devices become more ubiquitous.