%!TEX root = ../paper.tex

\begin{abstract}
\indent\indent Wearable devices bring great benefits but also potential risks that could expose users' activities without their awareness or consent. Effective design of notifications and security controls for wearable devices require careful foresight to prevent habituation and informing users about risks with which they are likely to be concerned. In this paper, we describe a large-scale survey conducted to investigate user security and privacy concerns regarding wearable devices. We surveyed 1,782 Internet users in order to identify risks that are particularly concerning to them; these risks are inspired by the capabilities of popular wearable technologies. During this experiment, our questions controlled for the effects of what data was being accessed and with whom it was being shared. We also investigated how these emergent threats compared to existent mobile threats, how upcoming capabilities and artifacts compared to existing technologies, and how users ranked technical and non-technical concerns to sketch a concrete and broad view of the wearable device landscape. We hope that this work will inform the design of future privacy and security controls for wearable devices.
\end{abstract}

% A category with the (minimum) three required fields
%\category{K.6.5.}{Management of Computing and Information Systems}{Security and protection}[Unauthorized access]

%\terms{Human Factors}{Measurement}{Security}

\keywords{Privacy, Security, User Studies, Risk Perception, Ubiquitous Computing, Wearable Devices} % NOT required for Proceedings