%!TEX root = ../paper.tex

\begin{abstract}
Along with great benefits, wearable devices, or ``wearables,'' bring new potential privacy and security risks which expose users' activities without their awareness or consent. With the additional capabilities of wearable devices and their increasing popularity, people have expressed interest in being notified before data capture \cite{denning2014situ}, but human attention is a finite resource \cite{bohme2011security}. Therefore, user concerns should be investigated to to warn users only about situations they are likely to care about. Informed, select notifications make for a better user experience and prevents habituation to such notifications while avoiding scandalous breaches of privacy. To this end, we conducted the first large-scale study to investigate user security and privacy concerns for wearable devices. We surveyed 1,784 Internet users for their perceptions of wearable devices and contribute: relevant perceived risks for wearables, effects of data type and data recipient on perceived risk, users' self reported concerns, and an assessment of how wearable device capabilities compare to familiar technologies. We conclude with a discussion on future research directions for wearable devices. 
\end{abstract}

% A category with the (minimum) three required fields
\category{K.6.5.}{Management of Computing and Information Systems}{Security and protection}[Unauthorized access]

%\terms{Human Factors}{Measurement}{Security}

\keywords{Privacy, Security, User Studies, Risk Perception, Ubiquitous Computing, Wearable Devices} % NOT required for Proceedings