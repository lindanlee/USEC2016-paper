%!TEX root = ../paper.tex

\begin{abstract}
Wearable devices bring great benefits but also new potential privacy and security risks that could expose users' activities without their awareness or consent. Effective design of notifications and security controls for wearable devices will require careful foresight to prevent habituation by treating user attention as a scarce resource, which requires understanding which risks have the potential to be most concerning to users. In this paper, we describe a large-scale survey that we conducted to investigate user security and privacy concerns regarding wearable devices. We surveyed 1,782 Internet users about their perceptions of wearable devices, in order to identify risks that are particularly concerning to users. We specifically controlled for the effects of data type and data recipient on the magnitude of perceived risks, while also collecting open-ended concerns. Finally, we compared wearable device risks to those of more familiar technologies. We hope that this work will inform design of future privacy and security controls for wearable devices.
\end{abstract}

% A category with the (minimum) three required fields
%\category{K.6.5.}{Management of Computing and Information Systems}{Security and protection}[Unauthorized access]

%\terms{Human Factors}{Measurement}{Security}

\keywords{Privacy, Security, User Studies, Risk Perception, Ubiquitous Computing, Wearable Devices} % NOT required for Proceedings