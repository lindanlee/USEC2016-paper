%!TEX root = ../paper.tex

\begin{abstract}
\indent\indent Wearable devices bring great benefits but also potential risks that could expose users' activities without their awareness or consent. Effective design of notifications and security controls for wearable devices require careful foresight to prevent habituation and informing users about risks which are the most concerning. In this paper, we describe a large-scale survey conducted to investigate user security and privacy concerns regarding wearable devices. We surveyed 1,782 Internet users in order to identify risks that are particularly concerning to users; these risks are inspired by the popular current wearable technologies' permissions and capabilities. During this experiment, we  determine the roles of the data being collected and with whom the data is being shared with through specifically controlling for these effects in our questions. We also investigate how these emergent threats compare to existent mobile threats, how upcoming capabilities and artifacts compare to existing technologies, and how users rank technical and non-technical concerns to sketch a concrete and broad view of the wearable device landscape. We hope that this work will inform design of future privacy and security controls for wearable devices.
\end{abstract}

% A category with the (minimum) three required fields
%\category{K.6.5.}{Management of Computing and Information Systems}{Security and protection}[Unauthorized access]

%\terms{Human Factors}{Measurement}{Security}

\keywords{Privacy, Security, User Studies, Risk Perception, Ubiquitous Computing, Wearable Devices} % NOT required for Proceedings