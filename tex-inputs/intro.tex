%!TEX root = ../paper.tex

\section{Introduction}

Wearables are a \$700 million, growing industry~\cite{cmo}. With 20\% of the general population owning at least one wearable and 10\% using it daily~\cite{WearableStatNews}, wearables are bringing ubiquitous computing to everyday life. This trend will likely continue, as 52\% of technology consumers are aware of wearables and 33\% are likely to buy one~\cite{NPD}.  

Wearable devices enable many benefits, ranging from interaction with virtual objects in an augmented reality world to healthier, fitness-data inspired lifestyles. However, wearable devices also bring new potential privacy and security risks that could expose users' activities without their awareness or consent. Although wearable devices are still in their infancy, we have already seen manifestations of these risks. Fitbit's default privacy settings inadvertently exposed information about some of their users' sexual activity~\cite{Fitbit}. Public discomfort toward facial recognition caused Google to prohibit Google Glass applications from using facial recognition~\cite{GlassDetection}, but still resulted in tech hate crimes against its users ~\cite{1_russell_2014, 16_gross_2014}. Google Glass has since been discontinued.

For smartphones, security and privacy risks are generally addressed by communicating data capture to users. However, many users are habituated to these notifications, because they see them frequently, often for things that they do not care about~\cite{felt2012android}. Once habituated to seemingly benign privacy and security warnings, users tend to ignore more sensitive warnings that are similarly designed~\cite{Egelman08}. 

Wearables' sensor capabilities, continuous access, and ubiquitous presence will result in a firehose of familiar and unfamiliar types of data, at rate which will likely dwarf the amount of data currently captured by smartphones. Bystanders of wearable devices have already expressed interest in such communication, desiring notification before data about them is captured~\cite{denning2014situ}. However, subjecting people with increased notifications is not a sound option, as it has shown to lead to negatively impactful effects such as frustration and habituation~\cite{bohme2011security}. An understanding of user concerns may allow for targeted and effective communication with the user, inform design of future permission systems, or provide insight for access control mechanisms. 

The goal of this work is to shape the still-malleable future of wearable platforms and interaction models, with research on user-centric concerns. To our knowledge, this is the first large-scale study to investigate user security and privacy concerns for wearable devices. Our survey of 1,782 Internet users contributes the following: %\\[-0.8cm]

\begin{itemize} \itemsep1pt \parskip0pt \parsep0pt
\item We report how 72 types of data likely to be captured by wearable devices were perceived by our participants and rank them by relevance. 
\item We repeat this across 4 possible recipients to also illustrate the contribution of the data recipient to the overall perceived risk. 
\item We sketch a landscape of users' self-reported concerns regarding wearable devices, spanning concerns outside of security and privacy. 
\item We compare emergent risks with existing risks and find that participants perceive risks similarly to physical risks---for instance, facial detection was perceived as risky as a lawnmower.
\end{itemize}
