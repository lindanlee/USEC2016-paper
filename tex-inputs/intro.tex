%!TEX root = ../paper.tex

\section{Introduction}

Wearable technologies, or ``wearables,'' are a \$700 million industry \cite{cmo} that may see significant growth in the near future: one market research company estimates that 52\% of technology consumers are aware of wearables and 33\% are likely to buy one~\cite{NPD}. With 20\% of the general population owning at least one wearable and 10\% using it daily~\cite{WearableStatNews}, wearables are bringing ubiquitous computing to everyday life.

% Forbes has named 2014 the ``Year of Wearable Technology \cite{Forbes}.''

Wearable devices offer many compelling opportunities for benefiting users.  For instance, they could enable more natural, human-centered interface experiences and healthier, fitness-data inspired lifestyles. There will likely be many more applications in the future that take advantage of data captured by wearable devices. However, wearable devices also bring new potential privacy and security risks that could expose users' activities without their awareness or consent. Even though wearable technology is currently in early stages, we have already seen manifestations of these risks. For instance, Fitbit's default privacy settings inadvertently exposed information about some of their users' sexual activity~\cite{Fitbit}, resulting in the inadvertent disclosure of sensitive information. Public discomfort toward facial recognition caused Google to prohibit Google Glass applications from using facial recognition~\cite{GlassDetection}, and public backlash may have partly contributed to Google's discontinuation of Glass~\cite{1_russell_2014, 16_gross_2014}.

Similar privacy and security issues pervade smartphones. Mobile platforms have tried to address the risks by communicating data capture to users. However, many users are habituated to these notifications, because they see them frequently, often for things that they do not care about~\cite{felt2012android}. Once habituated to seemingly benign privacy and security warnings, users tend to ignore more sensitive warnings that are similarly designed~\cite{Egelman08}.

With the additional capabilities of wearable devices and their increasing popularity, people have expressed interest in being notified before data is captured \cite{denning2014situ}. However, the possible data types that these devices will be able to capture is likely to dwarf those currently captured by smartphones. Because human attention is a finite resource \cite{bohme2011security}, bombarding users with notifications for every conceivable data capture is likely to be counter-productive as it can lead to user frustration and habituation. Therefore, user concerns need to be investigated so that systems can warn users only about the situations they are likely to care about.

The goal of this work is to gain a better sense of users' privacy concerns when interacting with wearable devices, so that future systems can take these concerns into account when designing privacy notifications. To our knowledge, this is the first large-scale study to investigate user security and privacy concerns for wearable devices. We surveyed 1,782 Internet users about their perceptions of wearable devices and contribute the following: %\\[-0.8cm]

\begin{itemize} \itemsep1pt \parskip0pt \parsep0pt
\item We compare users' perceptions of a range of privacy and security risks of wearables. We found that user concern is driven more by the type of data being captured than the recipient of the data.
\item We shed light on how users feel about various data recipients. We observed that users make less distinction between sharing data with friends, co-workers, and the general public, compared to sharing with an application's servers (the latter is viewed as less concerning).
\item We analyze users' self-reported top concerns regarding wearable devices. The results give a sense of what broad concerns users have, which can be used to guide future research on unexplored use cases. 
\item We examine how serious users perceive the risks of wearables to be, compared to risks associated with more familiar technologies. Most of our participants saw wearables' new capabilities as benign, though we suspect that this could be due to a lack of exposure to these newer technologies.
\end{itemize}
