%!TEX root = ../paper.tex

\section{Introduction}

Wearable technologies, or ``wearables,'' are a \$700 million industry with a promising future~\cite{cmo}. Recent surveys indicate that 52\% of technology consumers are aware of wearables and 33\% are likely to buy one~\cite{NPD}. With 20\% of the general population owning at least one wearable and 10\% using it daily~\cite{WearableStatNews}, wearables are bringing ubiquitous computing to everyday life. Wearable devices enable many benefits, ranging from interaction with virtual objects in an augmented reality world to healthier, fitness-data inspired lifestyles. 

However, wearable devices also bring new potential privacy and security risks that could expose users' activities without their awareness or consent. Although wearable devices are still in their infancy, we have already seen manifestations of these risks. Fitbit's default privacy settings inadvertently exposed information about some of their users' sexual activity~\cite{Fitbit}. Public discomfort toward facial recognition caused Google to prohibit Google Glass applications from using facial recognition~\cite{GlassDetection}; public backlash due to privacy concerns may have contributed to Google's discontinuation of Glass~\cite{1_russell_2014, 16_gross_2014}.

Similar issues pervade mobile platforms.  Generally, risks are addressed by communicating data capture to users. However, many users are habituated to these notifications, because they see them frequently, often for things that they do not care about~\cite{felt2012android}. Once habituated to seemingly benign privacy and security warnings, users tend to ignore more sensitive warnings that are similarly designed~\cite{Egelman08}.

The data captured by powerful, ubiquitous, and increasingly popular wearables will dwarf data currently captured by smartphones. It is known that subjecting users with increased notifications for every conceivable data capture has shown to be negatively impactful on security, as it can lead to user frustration and habituation~\cite{bohme2011security}. An understanding of user concerns allow for targeted, effective, and non-compromising communication with the user. Bystanders of wearable devices have already expressed interest in such communication, desiring notification before data about them is captured~\cite{denning2014situ}. Our study focuses on the users of wearable devices rather than bystanders, and explores what users would like notifications about on a large scale. 

The goal of this work is to shape the still-malleable future of wearable platforms and interaction models, with research on user-centric concerns. To our knowledge, this is the first large-scale study to investigate user security and privacy concerns for wearable devices. Our survey of 1,782 Internet users contributes the following: %\\[-0.8cm]

\begin{itemize} \itemsep1pt \parskip0pt \parsep0pt
\item We report how 72 types of data likely to be captured by wearable devices were perceived by our participants and rank them by relevance. 
\item We repeat this across 4 possible recipients to also illustrate the contribution of the data recipient to the overall perceived risk. 
\item We sketch a landscape of users' self-reported concerns regarding wearable devices, spanning concerns outside of security and privacy. 
\item We compare emergent risks with existing risks and find that participants perceive risks similarly to physical risks---for instance, facial detection was perceived as risky as a lawnmower.
\end{itemize}
