%!TEX root = ../paper.tex

\section{Introduction}

Wearable technologies, or ``wearables,'' are a \$700 million industry \cite{cmo} of electronically enhanced clothing items and accessories that interweaves technological interaction with everyday life. A top 25 market research company estimates that 52\% of technology consumers are aware of wearables and 33\% said they were likely to buy one~\cite{NPD}. With 20\% of the general population owning at least one wearable and 10\% using it daily~\cite{WearableStatNews}, wearables are transforming ubiquitous computing into a part of every day life. Forbes has named 2014 the ``Year of Wearable Technology \cite{Forbes}.''

The constantly captured data from these devices has many benefits, ranging from a more natural, human-centered interface experience to a healthier, fitness-data inspired lifestyle. There will likely be many more applications in the future which take advantage of such data. It is clear why wearable devices are becoming even more popular, especially as they have more capabilities and benefits over traditional devices. 

Along with these benefits, wearable devices bring new potential privacy and security risks which expose users' activities without their awareness or consent. Fitbit allowed sex to be tracked as exercise while fitness profiles were public by default~\cite{Fitbit}, resulting in the inadvertent disclosure of sensitive information. Public discomfort toward facial recognition prevented the capability from being deployed to Google Glass \cite{GlassDetection}. Google Glass, the iconic wearable of its time, has since disappeared \cite{13_google_2015}. Most suspect that the reason for its disappearance was because it was not a shift in interest, but public concern over privacy issues \cite{14_dvorak_2014}. Some Glass wearers faced assault \cite{1_russell_2014, 15_mashable_2014, 16_gross_2014} from uncomfortable bystanders.

We have seen similar privacy \cite{kelley2013privacy, sadeh2009understanding, shklovski2014leakiness} and security issues \cite{enck2011study, felt2011survey} related to data capture with respect to smartphones. Mobile platforms have tried to address this by communicating data capture to users as the data is captured. However, many users are habituated to these notifications, because they see them all the time, often for things that they don't care about \cite{felt2012android}.

With the additional capabilities of wearable devices and their increasing popularity, people have expressed interest in being notified before data capture \cite{denning2014situ}, but human attention is a finite resource \cite{bohme2011security}. Therefore, user concerns should be investigated to warn users only about situations they are likely to care about. Informed, select notifications make for a better user experience and prevents habituation to such notifications while avoiding scandalous breaches of privacy.

The goal of this paper is to gain a better sense of user concerns for wearables. To our knowledge, this is the first large-scale study to investigate user security and privacy concerns for wearable devices. We surveyed 1,784 Internet users for their perceptions of wearable devices and contribute the following: \\[-0.8cm]

\begin{itemize} \itemsep1pt \parskip0pt \parsep0pt
\item Comparisons of users' perceptions of a range of privacy and security risks of wearables. We found that users care much more about the type of data than the recipient of the data.
\item Insight into how users feel about various data recipients. We observed that users make less distinction between sharing data with friends, co-workers, and the general public, comparatively to sharing with an application's servers.
\item A report and categorization of users' self-reported top concerns for wearable devices. We give an sense of what broad concerns users have, which can be used to guide research in unexplored use cases. 
\item Rankings of how data-collection capabilities of wearable devices compared to more familiar technologies.  Most saw new capabilities as benign, but we suspect that this may be due to a lack of exposure to these newer technologies.
\end{itemize}