%!TEX root = ../paper.tex

\section{Introduction}

Wearable technologies, or ``wearables,'' are intimately related with both fields of ubiquitous computing and wearable computers. This \$700 million industry \cite{cmo} of electronically enhanced clothing items and accessories aims to interweave technological interaction into everyday life. A top 25 market research company estimates that 52\% of technology consumers are aware of wearables and 33\% said they were likely to buy one~\cite{NPD}. With 20\% of the general population owning at least one wearable and 10\% using it daily~\cite{WearableStatNews}, wearables are transforming ubiquitous computing into a part of every day life. Forbes has named 2014 the ``Year of Wearable Technology \cite{Forbes}.''

Because interactions with the wearables so frictionless, they are a popular choice for augmenting life with novel feedback and taking advantage of pre-existing applications more seamlessly. The constantly captured data from these devices has many benefits, ranging form a more natural, human-centered interface experience to healthier, fitness-data inspired lifestyle. It is clear why wearable devices are becoming even more popular, especially as they have more capabilities and benefits. 

Despite their popularity, the market is still in its infancy. It is hard to predict what device and applications will be most popular. Therefore, we use the current interest as a baseline. The most common wearables at the time of this study were in the form of fitness trackers and smart watches, with a survey of 3,956 wearables-enthusiastic respondents reporting the most popular devices as fitness bands (61\%), smart watches (45\%) and other health devices (17\%)~\cite{Nilsen}.  However, there are also wearable glasses \cite{ 2_google_2014, 3_sony_global_2014}, rings \cite{4_ringly_2014}, bracelets \cite{5_intel_2014} and more. New wearables trends for 2015 include smart clothes made out fiber-optic fabrics which give information about workouts and UV technology hairslide which measures sun exposure \cite{1_digital_trends_2014, 2_arthur_2014}.

Wearable devices bring with them new potential security and privacy concerns. Many concerns expose users' activities without their awareness or consent. For instance, Fitbit's fitness profiles were public by default and also allowed sex to be tracked as exercise~\cite{Fitbit}, resulting in the inadvertent disclosure of sensitive information. Additionally, public discomfort prevented Google from enabling capabilities such as facial detection on Google Glass \cite{GlassDetection}.

Although Google Glass was most iconic wearable of its time, it has since disappeared due to complaints from the public\cite{13_google_2015}. No formal statement from Google addresses the states that the reason for its disappearance is due to negative social pressures. However, most suspect that this is the reason for its disappearance, rather than a shift in interest. One popular tech magazine called glass the ``straw the broke the back of the privacy camel's back \cite{14_dvorak_2014},'' and made bystanders feel uncomfortable with the possibility of recording personal details, conversations, and photos. Many instances of assault \cite{1_russell_2014, 15_mashable_2014, 16_gross_2014} on wearers were reported, resulting from a combination of privacy concerns and social tension\footnote{Google has come to represent gentrification in San Francisco ... [which] makes wearable device owners look and sound like one of the people whom residents of the city have come to feel oppressed by \cite{17_matyszczyk_2014}. The term ``Glasshole'' \cite{18_google_2015} has been defined specifically to refer to Google glass wearers . The public was generally unsympathetic to the assults aggravated by social tension. Tweeting and comments included: ``I'm going to shove your pathetic nerd body into a locker,'' ``Nothing radicalizes you like having [your] home taken from you.'' and``As far as your 'hate crime' allegation is concerned, Entitled Tech Trash isn't a protected class just yet'' \cite{1_russell_2014,15_mashable_2014}.}. 

{\color {red} This calls for privacy research in this area! It's more important than ever! Or something along those lines.}  To avoid scandalous breaches of privacy and public opposition to new capabilities, it is critical we understand user concerns surrounding wearable devices to prevent negative social impact and breaches of privacy. There has been extensive work on the need for showing users and bystanders how certain data is being captured so that they can have more control. We have an issue of showing everyone everything all the time, since attention is a finite resources. We need to identify situations that people want to warn them about, and transparently access things that are benign. A better understanding of users' risk perceptions will enable researchers and companies to focus on users' concerns. The goal of this paper is to gain a better sense of what those concerns are. 

To obtain a comprehensive list of possible risks that wearable devices might present in the future, we examined the sensors, capabilities, permissions, and applications of the most popular wearable devices on the market at the time of this study. At the time of this study, August 2014, the most popular wearable devices included the fitbit fitness tracker which performs continuously monitors heartbeat, steps taken, and sleep patterns \cite{6_fitbit_2014, 7_time_2014}, the pebble smartwatch which can take pictures, send texts, show notifications from online, and push notifications to services \cite{pebble_smartwatch_2014, 9_verge_2014, 10_readwrite_2014}, and google glass \cite{11_wikipedia_2015, 12_turi_2014}. These wearable devices, along with other comparable wearable devices on the market, were researched as inspiration for the survey questions. 

{\color {red} better intro to here} We surveyed 1,784 Internet users for their perceptions of wearable devices. In this work, we contribute the following: \\[-0.8cm]

\begin{itemize} \itemsep1pt \parskip0pt \parsep0pt
\item We compare users' perceptions of a range of privacy and security risks of wearables. Users care much more about the type of data, than the recipient of the data.
\item We observed that users make little distinction between sharing data with friends, co-workers, or the general public, but are relatively comfortable with an application's servers receiving their data.
\item Our participants viewed data-collection capabilities of wearable devices as benign compared to more familiar technologies. However, we suspect that this may be due to a lack of exposure to these newer technologies.
%\item We report people's self-reported top concerns for wearable devices. Privacy, by far, is the top concern. Other notable risks include information security, long-term health risks from use, high financial cost, and change in social norms. 
\end{itemize}