%!TEX root = ../paper.tex

\section{Introduction}

Wearable technologies, or ``wearables,'' are a \$700 million industry \cite{cmo} consisting of electronically enhanced clothing items and accessories aims to interweave technological interaction into everyday life. A top 25 market research company estimates that 52\% of technology consumers are aware of wearables and 33\% said they were likely to buy one~\cite{NPD}. With 20\% of the general population owning at least one wearable and 10\% using it daily~\cite{WearableStatNews}, wearables are transforming ubiquitous computing into a part of every day life. Forbes has named 2014 the ``Year of Wearable Technology \cite{Forbes}.''

The constantly captured data from these devices has many benefits, ranging from a more natural, human-centered interface experience to a healthier, fitness-data inspired lifestyle. There will likely be many more applications in the future which take advantage of such data. It is clear why wearable devices are becoming even more popular, especially as they have more capabilities and benefits over traditional devices. 

Along with these benefits, wearable devices bring with them new potential privacy issues which expose users' activities without their awareness or consent. Fitbit allowed sex to be tracked as exercise while fitness profiles were public by default~\cite{Fitbit}, resulting in the inadvertent disclosure of sensitive information. Public discomfort toward facial recognition was so strong that the capability was not even deployed for Google Glass \cite{GlassDetection}. Google Glass, an iconic wearable of today, has since disappeared \cite{13_google_2015}. Most suspect that the reason for its disappearance was because it was the ``straw the broke the back of the privacy camel's back'' \cite{14_dvorak_2014}. Some Glass wearers faced assault \cite{1_russell_2014, 15_mashable_2014, 16_gross_2014}, while bystanders felt uncomfortable personal details, conversations, and photos possibly recorded.

{\color {red} smart phone concerns. we haven't gotten it right for this yet! We need to do something for wearables before it's too late.}

{\color {red} We should warn people about such things. But we can't do that, since attention is a finite resource. This calls for privacy research in this area!}  To avoid scandalous breaches of privacy and public opposition to new capabilities, it is critical we understand user concerns surrounding wearable devices to prevent negative social impact and breaches of privacy. There has been extensive work on the need for showing users and bystanders how certain data is being captured so that they can have more control. We have an issue of showing everyone everything all the time, since attention is a finite resource. We need to identify situations that people want to warn them about, and transparently access things that are benign. A better understanding of users' risk perceptions will enable researchers and companies to focus on users' concerns. The goal of this paper is to gain a better sense of what those concerns are. 

{\color {red} better intro to here} We surveyed 1,784 Internet users for their perceptions of wearable devices. In this work, we contribute the following: \\[-0.8cm]

\begin{itemize} \itemsep1pt \parskip0pt \parsep0pt
\item We compare users' perceptions of a range of privacy and security risks of wearables. Users care much more about the type of data, than the recipient of the data.
\item We observed that users make little distinction between sharing data with friends, co-workers, or the general public, but are relatively comfortable with an application's servers receiving their data.
\item Our participants viewed data-collection capabilities of wearable devices as benign compared to more familiar technologies. However, we suspect that this may be due to a lack of exposure to these newer technologies.
%\item We report people's self-reported top concerns for wearable devices. Privacy, by far, is the top concern. Other notable risks include information security, long-term health risks from use, high financial cost, and change in social norms. 
\end{itemize}