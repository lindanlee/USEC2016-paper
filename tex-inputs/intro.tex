\section{Introduction}
With their ability to constantly capture data and help users, wearable devices, or ``wearables,'' are the new frontier of ubiquitous computing. Wearable technology has many potential benefits, ranging from a more natural, human-centered interface for computing, to healthier living through fitness tracking. Forbes has named 2014 the ``Year of Wearable Technology''~\cite{Forbes}, and a top 25 market research company estimates that 52\% of technology consumers are aware of wearables and 33\% said they were likely to buy one~\cite{NPD}. 

The market for wearable devices is currently in infancy. It is hard to predict what device and applications will be most popular, but we use the current interest as a baseline. A survey of 3,956 respondents with high interest in wearables found that, currently, the most popular devices are fitness bands (61\%), followed by smart watches (45\%) and mobile health devices (17\%)~\cite{Nilsen}. It is estimated that 20\% of the general population owns at least one wearable and 10\% use at least one in their daily lives~\cite{WearableStatNews}. Most wearables consumers are young (48\% are between 18 and 34), but this \$700 million industry will reach other demographics soon~\cite{cmo}. 

Wearable devices bring with them new potential security and privacy concerns. Many concerns expose users' activities without their awareness or consent. For instance, Fitbit's fitness profiles were public by default and also allowed sex to be tracked as exercise~\cite{Fitbit}, resulting in the inadvertent disclosure of sensitive information. Additionally, public discomfort prevented companies from enabling capabilities; Google Glass apps are prohibited from using facial recognition to mitigate potential privacy concerns~\cite{GlassDetection}.

To avoid scandalous breaches of privacy and public opposition to new capabilities, it is critical we understand user concerns surrounding wearable devices before wearables become increasingly powerful and ubiquitous~\cite{Implants}. A better understanding of users' risk perceptions will enable researchers and companies to focus on users' concerns. The goal of this paper is to gain a better sense of what those concerns are. 

We surveyed 1,784 Internet users for their perceptions of wearable devices. In this work, we contribute the following: \\[-0.8cm]

\begin{itemize} \itemsep1pt \parskip0pt \parsep0pt
\item We compare users' perceptions of a range of privacy and security risks of wearables. Users care much more about the type of data, than the recipient of the data.
\item We observed that users make little distinction between sharing data with friends, co-workers, or the general public, but are relatively comfortable with an application's servers receiving their data.
\item Our participants viewed data-collection capabilities of wearable devices as benign compared to more familiar technologies. However, we suspect that this may be due to a lack of exposure to these newer technologies.
%\item We report people's self-reported top concerns for wearable devices. Privacy, by far, is the top concern. Other notable risks include information security, long-term health risks from use, high financial cost, and change in social norms. 
\end{itemize}