%!TEX root = ../paper.tex
\subsubsection{Demographic Factors}

We see that privacy is a main concern for wearables users. Additionally, we show that privacy preferences should also be a consideration for configuring a user's device. A participant's self-reported level of privacy concerns is the biggest demographic predictor of participants' VUR rate. This is determined by the IUIPC scale~\cite{malhotra2004internet}. A Spearman correlation yielded a statistically significant effect between average IUIPC scores and VUR ($\rho=0.446$, $p<0.0005$), which suggests responses to questions were mostly based on privacy preferences. Additionally, we observed that age was another significant predictor of VUR ($\rho=0.121$, $p<0.0005$), but we suspect that this effect is due to the significant correlation between age and IUIPC scores ($\rho=0.188$, $p<0.0005$). Others have observed that older individuals tend to be more privacy protective~\cite{varian2005demographics}.

While we initially observed an effect on VURs based on whether or not participants claimed to already own wearables (57.0\% vs. 60.8\%, respectively; Mann-Whitney $U=202,896$, $p<0.032$), this difference did not remain significant upon correcting for multiple testing (Bonferroni corrected $\alpha=0.01$). The effect of a participant's gender also did not remain significant upon correcting for multiple testing. We observed no correlation between a participant's education level and VUR.