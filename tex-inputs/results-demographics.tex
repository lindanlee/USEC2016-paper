%!TEX root = ../paper.tex
\subsubsection{Demographic Factors}

Participants' responses were correlated with demographic factors. We observed that the biggest predictor of participants' decisions to rate a scenario as very upsetting was their self-reported level of general privacy concerns, as determined by the IUIPC scale~\cite{malhotra2004internet}. A Spearman correlation yielded a statistically significant effect between average IUIPC scores with the VUR ($\rho=0.446$, $p<0.0005$), which suggests that responses to questions were mostly based on privacy preferences. Additionally, we observed that age was a significant predictor of VUR ($\rho=0.121$, $p<0.0005$). We suspect that the effect of age is due to the significant correlation between age and IUIPC scores ($\rho=0.188$, $p<0.0005$); others have observed that older individuals tend to be more protective of their privacy~\cite{varian2005demographics}.

While we initially observed an effect on VURs based on whether or not participants claimed to already own wearable devices (57.0\% vs. 60.8\%, respectively; Mann-Whitney $U=202,896$, $p<0.032$), this difference did not remain significant upon correcting for multiple testing (Bonferroni corrected $\alpha=0.01$). The effect of a participant's gender also did not remain significant upon correcting for multiple testing. We observed no correlation between a participant's education level and VUR.