%!TEX root = ../paper.tex

\section{Conclusion}

 Our survey of 1,784 Internet users, is the first large-scale study to investigate user-centric security and privacy concerns for wearable devices. We contribute a comprehensive ranking of possible risks associated with wearable devices, across various recipients. We calibrate our results with mobile devices and existing technologies; additionally, we verify with an open-ended question and find that privacy and security are at the top of user's overall concerns. Wearables are still in their infancy. Perceptions of situations and capabilities will change rapidly with advancements and increased exposure. However, there is not much work done to determine which threats in the emerging threat landscape are pertinent to focus on. Inspection of possible data concerns agree with previous studies of smartphone studies to find that video capture and financial data to be most sensitive. Various systems which detect and take actions for sensitive objects in photos and videos will be critical as wearables and other devices become more ubiquitous. We also find that users' self-reported privacy preferences are correlated with how participants may react, even with respect to situations that they are unfamiliar with. Permissions and access control mechanisms which do not depend on user inputs can still benefit from being informed by user preferences. We hope that this work has given a comprehensive overview of user concerns and inform designs of future privacy and security work for wearable devices. 