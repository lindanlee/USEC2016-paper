%!TEX root = ../paper.tex

\section{Conclusion}

Our survey of 1,784 Internet users for their perceptions of wearable devices was the first large-scale study to investigate user security and privacy concerns for wearable devices. We identify that participants are most concerned about protecting their financial information, photos, or videos and least concerned about demographic or biometric data. We find that participants have a significant difference in perception of risk when data is being shared with an application's server versus other human recipients, and also that certain data is more comfortably shared with certain recipients. Participants' self-reported concerns of new wearable device capabilities are also presented to give a sense of the relevant user concerns, which can be used to guide future research in privacy with respect to wearable devices, especially warnings, notifications, and permissions. We perform two regression model analyses with respect to the type of data and the associated risks to verify our results.

%commented out for anonymity

%\section{Research Ethics} 
%We received advance approval from University of California, Berkeley's Institutional Review Board to perform this user study. Survey data was collected in an anonymous manner. Although the focus group was not conducted in an anonymous manner, participants were not asked to offer any confidential or sensitive information. 

%\section{Acknowledgments}
%The authors would like to thank The National Science Foundation's Graduate Research Fellowships Program (NSF GRFP), and the Intel Science and Technology Center for Secure Computing (also known as  Secure Computing Research for Users' Benefit, SCRUB), for the support of this research. Inspiration for this work emerged out of fruitful discussions at the Berkeley Laboratory for Usable and Experimental Security (BLUES) meetings. Many thanks go out to the many generous colleagues who have provided feedback and encouragement.