\subsubsection{Data Type}
Based on our data, we observed that the largest effect stemmed from the data type being shared in a scenario. We present various statistical models in section \ref{sec:regression} to support this conclusion. The 10 most and 10 least concerning data types can be seen in Table \ref{top10}. 

Regardless of the data recipient or the device, participants were most concerned about photos and videos, especially if they contained embarrassing content, nudity, or financial information. As seen in Table \ref{top10}, photos and videos accounted for 5 of the top 10 concerns. Information that could be used to impersonate someone (e.g., usernames/passwords for websites) or invade privacy (photos of someone at home) were also among the most concerning data types. 

Also regardless of the data recipient, the least concerning data types mostly consisted of information that could be observed through observations of public behavior, such as demographic information (e.g., age, gender, language spoken). It is possible that people rated these as unconcerning because they think many entities already track this data (e.g., shows watched, music listened to, exercise patterns).

{\color {red} talk about the variance of the data types here, referring to the table in the appendix. Which were the highest in variance? Which were the unanimous? Which one had a spread? Were there any that were specifically polarized?}

\begin{table}[t]
\begin{center}
\small
\begin{tabular}{| r | l | r |}
\hline
Rank & Data &  VUR  \\
\hline
1 & a video of you unclothed & 95.97\% \\
2 & bank account information & 95.91\% \\
3 & social security number & 94.84\% \\
4 & video of you entering in your PIN & 92.67\% \\
5 & a photo of you unclothed & 92.59\% \\
6 & an incriminating/embarrassing photo of you & 91.39\% \\
7 & username and password for websites & 89.55\% \\
8 & credit card information & 88.98\% \\
9 & an incriminating/embarrassing video of you & 88.41\% \\
10 & a random (inward-facing) photo you at home & 87.50\% \\
 & \vdots & \\
64 & eye movement patterns (for eye tracking) & 40.51\% \\
65 & when and how much you exercise  & 38.66\% \\
66 & when you are happy or having fun  & 34.75\% \\
67 & which television shows you watch & 30.20\% \\
68 & when you are busy or interruptible  & 29.50\% \\
69 & music from your device  & 28.06\% \\
70 & your heart rate & 27.50\% \\
71 & your age & 24.29\% \\
72 & the language you speak & 15.86\% \\
73 & your gender & 15.00\% \\ 
\hline
\end{tabular}
\caption{The 10 most and least upsetting data types, across all recipients.}
\label{top10}
\end{center}
\end{table}

%A statistical analysis regarding the significance and confidence of <data> types with respect to all 72 was not performed due to the space constraints of the paper. We do consider all <data> categories in our statistical model, which provides an analysis of what factors had contributed to the perceived severity of a particular situation. 