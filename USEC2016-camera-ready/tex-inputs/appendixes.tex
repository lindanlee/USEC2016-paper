%!TEX root = ../camera-ready.tex
\appendix

%\section{Fischhoff Prompts}
%\label{sec:prompt} 
%
%\textit{We would like to ask you to rate the <risks/benefits> associated with each of the following technologies.}
%
%{\bf Risks:} \textit{Consider all types of risks: the risk of physical harm or death, the risk to others or bystanders, the financial cost of the technology, any distress caused by the technology, what the consequences would be if the technology was misused, any impact on the public, work, or personal life, and other considerations. (e.g. for electricity, consider the risk of electrocution, the pollution caused by coal, the risk that miners need to take to mine the coal, the cost to build the infrastructure to deliver electricity, etc.) Give a global estimate over a long period of time (say, a year) of both intangible and tangible risks.} 
%
%\textit{Do not consider the costs or risks associated with these items. It is true, for example, that sometimes swimmers can drown. But evaluating such risks is not your present job. Your job is to assess the gross benefits, not the net benefits which remain after the costs and risks are subtracted out.}
%
%\textit{Please rate the following technologies below with a number. We know that this might be a bit hard to do, but please try to be as accurate as possible, adjusting the numbers until they feel they are right for you. Start with the least risky technology at 10 and assign higher numbers for the more risky technologies. (For instance, a technology rated 14 is half as risky as a technology rated 28.)}
%
%{\bf Benefits:} \textit{Consider all types of benefits: how many jobs are created, how much money is generated directly or indirectly, how much enjoyment is brought to people, how much a contribution is made to the people's health and welfare, what this technology promotes, and so on. (e.g. for swimming, consider the manufacture and sale of swimsuits, the time spent exercising, the social interactions during swimming, and the sport created around the activity.) Give a global estimate over a long period of time (say, a year) of both intangible and tangible benefits.}
%
%\textit{Do not consider the costs or benefits associated with these items. It is true, for example, that electricity also creates a market for home appliances. But evaluating such benefits is not your present job. Your job is to assess the gross risks, not the net risks which remain after the costs and risks are subtracted out.} 
%
%\textit{Please rate the following technologies below with a number. We know that this might be a bit hard to do, but please try to be as accurate as possible, adjusting the numbers until they feel they are right for you. Start with the least beneficial technology at 10 and assign higher numbers for the more beneficial technologies. (For instance, a technology rated 34 is twice as beneficial as a technology rated 17.)}

\noindent {\bfseries Omitted Survey Questions}
\label{sec:noquestions}
Our participants answered four additional question which are not detailed in this paper. Two questions compared smartphones to wearable devices, to investigate if participants inherently saw wearables as threatening due to its unfamiliarity. The questions were not particularly well suited for this task, and our results showed no significant difference between smartphone and wearables perceptions. 

The other two questions asked our participants to numerically rate the risks and benefits of common, familiar technologies versus wearables (e.g., an airplane versus a wearable). This was to mimic the methodology in Fischhoff's seminal study on attitudes toward technological risks and benefits.In Fischhoff's study, all technologies were concerning with physical risk, while ours mixed information disclosure risks with physical risks. These comparisons were unsound to make. Nonetheless, our results largely highlighted our participants' unfamiliarity with wearables or other less common devices; generally, participants rated more familiar technologies as more beneficial or risky. \\


\noindent {\bfseries Coding Label Definitions}
\label{sec:coding}
We realize that coding answers ranging from 1 to 1,400+ words into a single, one-word category is tricky. To give some transparency into how we coded the answers, we provide the coding outline we used. If an answer mentioned multiple concerns, the answer was given multiple coding labels\\

\noindent {\bf Privacy}: ``privacy,'' mention of personal details, spying. \\
{\bf Security}:  ``security,'' mention of malware, hacking. \\
{\bf GPS tracking}: ``location,'' ``GPS,'' mention of monitoring. \\
{\bf Being Unaware}: mention of using, collecting, and disclosing data without permission. \\
{\bf False information}: inaccurate or maliciously false data.\\
{\bf Health Risk}: mention of radiation, cancer, or other effects.\\
{\bf Safety}: mention of distractions causing car crashes and injuries, violence due to the device, injuries from malfunctions.\\
{\bf Discomfort}: mention of eye strain, headache, irritation. \\
{\bf Financial cost}: cost of buying or using the device. \\
{\bf Theft}: mention of device theft. \\
{\bf Social Impact}: mention of dependency, distance from people, changes in decision making, etc. \\
{\bf Social Stigma}: mention of judgment, hate, or bystanders.\\ 
{\bf Aesthetics}: mention of fashion or looking dorky. \\
{\bf Miscellaneous}: odd comments, uncommon concerns. \\
{\bf None}: ``None,'' mention of no threat, or no real concerns \\
{\bf Don't know}: ``do not know,'' general confusion \\
{\bf Don't care}: `` do not care,'' nonchalant answers 

