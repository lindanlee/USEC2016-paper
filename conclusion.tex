\section{Conclusion}

We surveyed 2,250 internet users to determine what contributes to a violation of privacy or security, which technologies are risky, and what users think are the biggest risk for operating wearable devices. Participants how upset they would be if 304 situations occurred, assessed the risk and benefit for 20 new technologies, and gave open-ended responses to express their concerns. We provide insight into how much and why data, recipient, and device contribute to users' perception of a situation, calibrate answers with existing smartphone literature, and provide a regression model. An assessment of a range of new technologies shows that users perceive new technologies to be low-risk and low-benefit, but we suspect this is due to limited exposure that an average person has with wearables techology. We also state what users perceived as the most significant concerns with respect to wearable devices. We conclude by discussing future research directions in the wearables and user study space. 

%commented out for anonymity

%\section{Research Ethics} 
%We received advance approval from University of California, Berkeley's Institutional Review Board to perform this user study. Survey data was collected in an anonymous manner. Although the focus group was not conducted in an anonymous manner, participants were not asked to offer any confidential or sensitive information. 

%\section{Acknowledgments}
%The authors would like to thank The National Science Foundation's Graduate Research Fellowships Program (NSF GRFP), and the Intel Science and Technology Center for Secure Computing (also known as  Secure Computing Research for Users' Benefit, SCRUB), for the support of this research. Inspiration for this work emerged out of fruitful discussions at the Berkeley Laboratory for Usable and Experimental Security (BLUES) meetings. Many thanks go out to the many generous colleagues who have provided feedback and encouragement.